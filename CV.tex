% Adapted from layout by GaelVaroquaux
%  http://gael-varoquaux.info
%

\documentclass{article} %{{{--

\usepackage[paper=a4paper,
	    top=2cm,
	    left=1.35cm,
	    width=18.2cm,
	    bottom=2cm
	    ]{geometry}
            %margin=4cm,

\usepackage{calc}
\usepackage[T1]{fontenc}
\usepackage[utf8]{inputenc}
\usepackage{lmodern}
\usepackage{color,hyperref}
\usepackage{graphicx}
\usepackage{multicol}

\usepackage{wasysym} % For phone symbol
\usepackage{url}

\def\bf{\bfseries}
\def\sf{\sffamily}
\def\sl{\slshape}
% Semi condensed bold

\definecolor{deep_blue}{rgb}{0,.2,.5}
\definecolor{dark_blue}{rgb}{0,.1,.3}
\definecolor{myblue}{rgb}{.01,0.21,0.71}
\definecolor{gray}{rgb}{.5, .5, .5}

\hypersetup{pdftex,  % needed for pdflatex
  breaklinks=true,  % so long urls are correctly broken across lines
  colorlinks=true,
  urlcolor=myblue,
  %linkcolor=darkblue,
  %citecolor=darkgreen,
  }


%% This gives us fun enumeration environments.
\usepackage{enumitem}

%% More layout: Get rid of indenting throughout entire document
\setlength{\parindent}{0in}

%% Reference the last page in the page number
%
\usepackage{fancyhdr,lastpage}
\pagestyle{fancy}
%\pagestyle{empty}      % Uncomment this to get rid of page numbers
\fancyhf{}\renewcommand{\headrulewidth}{0pt}
%\fancyfootoffset{\marginparsep}+\marginparwidth}
\lfoot{
  \hspace{-2\marginparsep}
  \,\hfill \arabic{page} of \protect\pageref*{LastPage}  \hfill \,\\
  \,\hfill {\footnotesize \textcolor{gray}{updated August 2016~~~~~}} \hfill\,
}

\newcommand{\mydate}[1]{{\textcolor{gray}{\footnotesize #1}}}


\newcommand{\makeheading}[1]%
        {%\hspace*{-\marginparsep minus \marginparwidth}%
         %\begin{minipage}[t]{\textwidth+\marginparwidth+\marginparsep}%
         \begin{minipage}[t]{\textwidth}%
                {\Large #1}\\%[-0.5\baselineskip]%
                \vskip 0.2\baselineskip
                 \color{deep_blue}{\rule{\columnwidth}{3pt}}%
         \end{minipage}
	 \vskip 1.\baselineskip plus 2\baselineskip minus 1.\baselineskip
	}

\newlength\sidebarwidth
\setlength\sidebarwidth{3.6cm}

\newcommand{\topic}[3][]%
	 {\pagebreak[2]%
	 \vskip 1.5\baselineskip plus 3\baselineskip minus 0.7\baselineskip
	 \begin{minipage}{\textwidth}
         \phantomsection\addcontentsline{toc}{section}{#1}%
         \nopagebreak\hspace{0in}%
         \nopagebreak\begin{minipage}[t]{\sidebarwidth - .2cm}
         \raggedleft \bf\sf
	 \color{deep_blue}{\Large #2}
	 \end{minipage}%
	 \hfill
	 \begin{minipage}[t]{\linewidth - \sidebarwidth}
	 \nopagebreak{\color{deep_blue}%
		    \rule{0pt}{\baselineskip}%
		    \rule{\linewidth}{2.5pt}%
		    \llap{\raisebox{.3\baselineskip}{\sf #1}}%
		    \vspace*{.1\baselineskip}%
		    }%
	 #3%
	 \end{minipage}
	 \end{minipage}}

\newcommand{\smalltopic}[2]%
	 {\pagebreak[2]%
	 \vskip 1\baselineskip plus 2\baselineskip minus 0.3\baselineskip
	 \begin{minipage}{\textwidth}
	 %\hspace{-\marginparsep minus \marginparwidth}%
         \phantomsection\addcontentsline{toc}{subsection}{#1}%
         \nopagebreak\hspace{0in}%
         \nopagebreak\begin{minipage}[t]{\sidebarwidth - .2cm}
         \raggedleft \bf\sf %\vskip -0.5\baselineskip
	 \textcolor{dark_blue}{\large #1}%
	 \end{minipage}%
	 \hfill
	 \begin{minipage}[t]{\linewidth - \sidebarwidth}
	 \nopagebreak{%
	    %\vspace{-.7\baselineskip}%
	    \rule{\linewidth}{.5pt}%
	    \vspace{.1\baselineskip}%
	    }%
	    #2
	 \end{minipage}
	 \end{minipage}}

\newcommand{\subtopic}[3][]
	 {\begin{minipage}{\textwidth}
	 \vspace*{.4\baselineskip}
         \nopagebreak\hspace{0in}%
         \nopagebreak\begin{minipage}[t]{\sidebarwidth - .2cm}
	 % Super posh: using semi-bold condensed fonts. Works only with
	 % lmodern
         \raggedleft {\sf\fontseries{sbc}\selectfont #2}
	 %{\small\sl\\[-0.2\baselineskip] #1}
         {\\[-0.2\baselineskip] \textcolor{gray}{\footnotesize #1}}
	 \end{minipage}%
	 \hfill
	 \begin{minipage}[t]{\linewidth - \sidebarwidth}
	 #3%
	 \end{minipage}%
	 \vspace*{.2\baselineskip plus 1\baselineskip minus
	 .2\baselineskip}%
	 \end{minipage}}

\newcommand{\dateonly}[2][]
	 {\begin{minipage}{\textwidth}
	 \vspace*{.4\baselineskip}
         \nopagebreak\hspace{0in}%
         \nopagebreak\begin{minipage}[t]{\sidebarwidth - .2cm}
         \raggedleft {~}
         {\\[-\baselineskip] \textcolor{gray}{\footnotesize #1}}
	 \end{minipage}%
	 \hfill
	 \begin{minipage}[t]{\linewidth - \sidebarwidth}
	 #2%
	 \end{minipage}%
	 \vspace*{.2\baselineskip plus 1\baselineskip minus
	 .2\baselineskip}%
	 \end{minipage}}

\newcommand{\sidenote}[2]
	 {\vspace*{-.2\baselineskip}\begin{minipage}{\textwidth}
         \nopagebreak\hspace{0in}%
         \nopagebreak\begin{minipage}[t]{\sidebarwidth - .2cm}
         \raggedleft {#1}
	 \end{minipage}%
	 \hfill
	 \begin{minipage}[t]{\linewidth - \sidebarwidth}
	 #2%
	 \end{minipage}%
	 \vspace*{.5\baselineskip plus 1\baselineskip minus
	 .2\baselineskip}%
	 \end{minipage}}

% New lists environments
\newlist{outerlist}{itemize}{1}
\setlist[outerlist]{font=\sffamily\bfseries, label=\textbullet}
\setitemize{topsep=0ex, partopsep=0ex}
\setdescription{font=\normalfont\sffamily\bfseries, itemsep=.5ex,
    parsep=.5ex, leftmargin=3ex}

\newcommand{\blankline}{\quad\pagebreak[2]}

\def\mydot{\textcolor{deep_blue}{\rule{1ex}{1ex}}}

%%%%%%%%%%%%%%%%%%%%%%%%%%%%%%%%%%%%%%%%%%%%%%%%%%%%%%%%%%%%%%%%%%%%%--}}}%
\begin{document}
\makeheading{
\begin{minipage}[B]{0.5\textwidth}
    %\vfill
    \vspace*{-.5\baselineskip}%
    \parbox{10cm}{
	\hskip -0.1cm
	{\Huge\bf\sf \color{deep_blue} D\huge \hskip -0.05cm ANIEL %
	 \Huge \color{deep_blue} B\huge \hskip -0.05cm USCOMBE}
	\\[-.1\baselineskip]
	{\bf\sf Assistant Research Professor}
        \\
        {\sf Northern Arizona University}
        
    }
\end{minipage}
\hfill
\begin{minipage}[B]{8cm}
    \raggedleft
    \,\vskip -1em
    \small
        School of Earth Sciences \& Environmental Sustainability\\
        525 S Beaver St., PO Box: 5694\\
        Flagstaff AZ 86011\\
	{\texttt {dbuscombe@usgs.gov}}%
    \vspace*{-.5\baselineskip}%
\end{minipage}
}

%\begin{center}
%\begin{minipage}{15cm}
\begin{multicols}{2}
\sloppy

%\textcolor{deep_blue}{\bf\sf Research interests}:
%Cosmology, weak lensing, data mining and automated learning algorithms for
%large astronomical data sets.

\begin{itemize}[leftmargin=2ex, itemsep=0ex]
\item[\mydot]
I am a Research Professor at NAU, teaching and researching in sedimentology, geomorphology, geostatistics, geophysics, and
scalable computation. 

\item[\mydot]
I am a contractor for the Grand Canyon Monitoring and Research Center, an
interdisciplinary program within the U.S. Geological Survey carrying out basic and applied research into management of regulated rivers in the Western United States.

\item[\mydot]
Previously I was an a post-doctoral fellow,
working in the Marine Science departments
at the University of Plymouth, and before that at the University of California Santa Cruz. During those fellowships, I worked on
sediment transport and hydrodynamics in nearshore and coastal shelf environments.

\item[\mydot]
My PhD research at the University of Plymouth (2004 - 2008) focused on the morphodynamics and sediment dynamics of macrotidal gravel beaches.

\item[\mydot]
I am interested in encouraging reproducible and open research practices
in geosciences.  To this end I spend significant time developing a variety of open-source
scientific computing tools.

\end{itemize}
\end{multicols}
\vspace*{-1.5em}
\fussy
%\end{minipage}
%\end{center}

%%%%%%%%%%%%%%%%%%%%%%%%%%%%%%%%%%%%%%%%%%%%%%%%%%%%%%%%%%%%%%%%%%%%%%%%%%%
\topic{E \large\hskip -1ex DUCATION}{~}

    \subtopic[2004-2008]{\bf PhD}{
        School of Geography, University of Plymouth, Plymouth, UK\\
        advised by Gerd Masselink and Mark Davidson (School of Marine Science and Engineering)\\
	Thesis: \href{https://dbuscombe-usgs.github.io/media/pdfs/Buscombe_thesis_june2008.pdf}{
          Morphodynamics, Sediment Dynamics and Sedimentation of a Gravel Beach}
    }

    \subtopic[2000-2003]{\bf BSc}{
      Bowland College, Lancaster University, Lancaster, UK\\
      Major: Physical Geography; Minors: Environmental Science \& Biological Science\\
			First class, with honours.
		}

%%%%%%%%%%%%%%%%%%%%%%%%%%%%%%%%%%%%%%%%%%%%%%%%%%%%%%%%%%%%%%%%%%%%%%%%%%%
\topic{E \large\hskip -1ex XPERIENCE}{~}

\vspace*{-0.5\baselineskip}
\smalltopic{Employment}{}

    \subtopic[Nov. 2016--Present]{NAU}{
      Assistant Research Professor.\\
      School of Earth Science \& Environmental Sustainability}

    \subtopic[2012-- Nov. 2016]{U.S. Geological Survey}{
      Research Geologist.\\
      Grand Canyon Monitoring \& Research Center, U.S. Geological Survey.\\
			Supervised by Dr Paul Grams.}

    \subtopic[2009--2012]{UoP Marine Science}{
      NERC post-doctoral fellowship.\\
      Department of Marine Science \& Engineering, University of Plymouth, UK.\\
      Supervised by Dr Daniel Conley \& Dr Alex Nimmo-Smith.}

    \subtopic[2008--2009]{UCSC Marine Science}{
      Postdoctoral Researcher, U.S. Geological Survey\\
      \& Department of Earth \& Planetary Sciences, University of California Santa Cruz.\\
      Supervised by Dr David Rubin \& Dr Jessica Lacy.}

    \subtopic[2008--2011]{UoP Marine Ecology}{
      Computer Programming Contractor.\\
			Marine Biology \& Ecology Research Centre, University of Plymouth, UK.\\
			Supervised by Dr Kerry Howell.}

    \subtopic[2008]{UoP BARDEX}{
		Research Assistant, School of Geography, University of Plymouth, UK.\\
		BARDEX (Barrier Dynamics Experiment), an EU Hydralab III-funded laboratory wave flume project.
		Supervised by Prof. Jon Williams \& Prof. Gerhard Masselink.
    }

		\subtopic[2007--2008]{UoP WHISSP}{
		Research Assistant, School of Marine Science \& Engineering, University of Plymouth, UK.\\
		WHISSP (Wave Hub Impacts on Seabed and Shoreline Processes), an EU-funded field-based project studying the effects of marine renewable energy devices on the shoreline.
		Supervised by Prof. Jon Williams \& Prof. Gerhard Masselink.
    }

    \subtopic[2004--2008]{UoP Geography}{
      Associate Lecturer and Demonstrator (part-time).\\
			School of Geography, University of Plymouth, UK.
    }

    \subtopic[2003--2004]{FSC}{
      Assistant tutor.\\
			Field Studies Council, Castle Head, Grange-over-Sands, UK.
    }


\smalltopic{Teaching}{}

    \subtopic[Spring 2017]{EES 680}{
      Earth \& Environmental Data Analysis\\
      {\it University of Northern Arizona}}

    \subtopic[Fall 2015]{EES 698-1}{
      Topics in Fluvial Geomorphology\\
      {\it guest lecturer -- University of Northern Arizona}}

    \subtopic[Fall 2014--2016]{EES 529}{
      Applied Remote Sensing\\
      {\it guest lecturer -- University of Northern Arizona}}

    \subtopic[2010 -- 2012]{OS204}{
      Waves, Tides and Coastal Dynamics\\
      {\it guest lecturer -- University of Plymouth}}

    \subtopic[2010 -- 2012]{OS311}{
      Modelling Ocean Processes\\
      {\it guest lecturer -- University of Plymouth}}

    \subtopic[2004--2008]{Geography}{
      Introductory statistics, Glacial Geomorphology, Coastal Geomorphology\\
      {\it teaching assistant -- University of Plymouth}}

    \subtopic[2003--2004]{FSC}{
      Field- and classroom-based ecology, geology, environmental sciences\\
      {\it teaching assistant -- Field Studies Council Castle Head}}


\smalltopic{Service}{}

  \subtopic[2007-Present]{Journal Reviewer}{
     Arctic; Continental Shelf Research; Earth Surface Processes \& Landforms; Geo-Marine Letters; Geophysical Research Letters; Journal of Hydraulic Engineering; Journal of Marine Science \& Engineering; Journal of Mountain Science; Journal of Sedimentary Research; Marine Geology; Sedimentology; Sedimentary Geology; Water Resources Research.
   }

   \subtopic[2016]{Software Carpentry}{
     Lead organizer for this 3 day-long, 30-person workshop at U.S. Geological Survey.
   }

   \subtopic[2015]{MBES in Rivers}{
     Lead organizer for 2nd Multibeam in Rivers Workshop, a 3 day-long, 30-person workshop at U.S. Geological Survey.
   }

   \subtopic[2013]{AGU}{
     Co-convener of the session, EP010. Fluvial sediment budgets: Can we do better? American Geophysical Union Fall Meeting, December 2013
   }

   \subtopic[2007]{AGU}{
     Co-convener of the session, H60: Linking sediment supply, bed-sediment particle size, sediment transport, and bed morphology in fluvial, marine, and aeolian settings. American Geophysical Union Fall Meeting, December 2007
   }

   \subtopic[2007]{YCSEC}{
     On the organising committee for the Young Coastal Scientist and Engineers Conference, 2007 (YCSEC 2007)  hosted by the School of Geography at the University of Plymouth 19-21 April 2007.
   }

   \subtopic[2005]{QRA}{
     On the organising committee for the Quaternary Research Association's 4th International Postgraduate Symposium, hosted by the School of Geography at the University of Plymouth 31st August - 2nd September 2005.
   }


\smalltopic{Mentoring}{}

    \subtopic[2016-Present]{Andrew Platt}{
	  MS, Northern Arizona University School of Earth Sciences \& Environmental Sustainability.\\
		Co-supervised with Dr Ryan Porter.
	}

    \subtopic[2016-Present]{Ryan Lima}{
	  PhD, Northern Arizona University School of Earth Sciences \& Environmental Sustainability.\\
		Co-supervised with Dr Temuulen Sankey.
	}

    \subtopic[2015-Present]{Daniel Hamill}{
      MS, Utah State University Department of Watershed Sciences.\\
			Co-supervised with Dr Joseph Wheaton.
    }

    \subtopic[2014-Present]{Rebecca Rossi}{
		MS, Utah State University Department of Watershed Sciences.\\
		Co-supervised with Dr Joseph Wheaton.
		}

    \subtopic[2014-Present]{Thomas Ashley}{
      PhD, School of Geology and Geophysics, University of Wyoming.\\
			Co-supervised with Dr Brandon McElroy.
    }

    \subtopic[2011--2012]{Martin Meoli}{
      MSc Applied Marine Science, School of Marine Science \& Engineering, University of Plymouth\\
			Co-supervised with Dr Alex Nimmo-Smith.
    }

    \subtopic[2011--2012]{James Sawyer}{
		MSc Applied Marine Science, School of Marine Science \& Engineering, University of Plymouth\\
		Co-supervised with Dr Daniel Conley.
    }


%%%%%%%%%%%%%%%%%%%%%%%%%%%%%%%%%%%%%%%%%%%%%%%%%%%%%%%%%%%%%%%%%%%%%%%%%%%
\topic{A \large\hskip -1ex WARDS \& HONORS}{~}

\subtopic[March 2016]{USGS "What's the Big Idea?""}{
	Research featured in the video \href{https://www.youtube.com/watch?v=zNqpuKf2O7o}{What's the Big Idea? —Using Sound to Remotely Sense the Riverbed}
on the YouTube channel of the U.S. Geological Survey
}

    \subtopic[July 2015]{American Geophysical Union Research Spotlight}{
      Research featured in the article \href{https://eos.org/research-spotlights/using-sound-waves-to-study-grand-canyon-sediment}{Using Sound Waves to Study Grand Canyon Sediment}
      in EOS Earth and Space Science News
    }

		\subtopic[2015]{Elsevier "Excellence in Peer Review" Award}{
      for the Elsevier journal, {\it Sedimentary Geology}
    }

    \subtopic[September 2012]{JGR-Oceans Editor's Highlight}{
      Research featured in the article \href{http://agupubs.onlinelibrary.wiley.com/agu/article/10.1029/2012JC007954/editor-highlight/}{Novel observations of currents and drag generated by a tsunami}
			published in the Journal of Geophysical Research - Oceans.
    }



%%%%%%%%%%%%%%%%%%%%%%%%%%%%%%%%%%%%%%%%%%%%%%%%%%%%%%%%%%%%%%%%%%%%%%%%%%%
\topic{C \large\hskip -1ex OMPUTING}{
  I am an active developer and maintainer of several
  scientific computing packages.
  See my github profile
  (\href{http://github.com/dbuscombe-usgs}{http://github.com/dbuscombe-usgs})
  for details.}

%\vspace*{-0.5\baselineskip}
\smalltopic{Skills}{

  \begin{itemize}[leftmargin=0ex, itemsep=0ex, labelindent=-2ex, parsep=.5ex]
    \item[\mydot] Experienced open source developer, with a specialization in
      scientific computing, including visualization, geospatial statistics, signal processing, image processing
      and machine learning.
    \item[\mydot] Expert in the Python Language and extensions such as Cython;
      expert in the MATLAB language, experience writing C, C++, and Fortran code.
    \item[\mydot] Experience with a variety of tools and languages, including
      bash, csh, \LaTeX{}, HTML, Git, Linux, virtual machines, virtual environments, auto-deployment of software packages (PyPI, SWIG, Distutils, conda), distributed, parallel, out-of-core and cloud computing.
    \item[\mydot] Experience with Hydrodynamic modelling software, including the \href{http://www.gotm.net/index.php}{General Ocean Turbulence Model}; Simulating Waves Nearshore (\href{http://www.swan.tudelft.nl/}{SWAN}); Simulating Waves 'til Shore (\href{http://swash.sourceforge.net/features/features.htm}{SWASH}).
    \item[\mydot] Experience with Hydrographic surveying and mapping software, including \href{http://gmt.soest.hawaii.edu/}{Generic Mapping Tools}, \href{http://www.ldeo.columbia.edu/res/pi/MB-System/}{MB-System} and \href{www.hypack.com}{HYPACK}, GIS, and geospatial libaries such as GDAL, Mapbox and Proj-4.
  \end{itemize}

}

\smalltopic{Major Software Projects}{}

   \subtopic[2010--Present]{DGS}{
     Software for automated analyses of grain size from images of sediment. Source code currently available in Matlab and Python.}

   \subtopic[2014--Present]{PyHum}{
  Software for reading, processing and analysis of Humminbird sidescan data. Source code available in Python/Cython.
   }

   \subtopic[2015--Present]{pysesa}{
Python program for spatially explicit spectral analysis. Software for spatially explicit analyis of point clouds and spatially distributed data. Source code available in Python. 
   }

   \subtopic[2011--2012]{Sand Simulation Toolbox}{
Software for generating 3D discrete particle models consisting of realistic particles (with a size- and shape-distribution) with user-defined properties. Source code available in Matlab. 
   }

   \subtopic[20012]{MATSCAT}{
Software for analysis of multiple-frequency acoustic backscatter for suspended sediment concentration and particle size. Source code available in Matlab.}

   \subtopic[2008--2011]{Benthic Analysis Tool}{
Software for the semi-automation of species identification and measurement in deep-sea ROV/drop frame images. Source code available in Matlab.
   }


\newpage

%%%%%%%%%%%%%%%%%%%%%%%%%%%%%%%%%%%%%%%%%%%%%%%%%%%%%%%%%%%%%%%%%%%%%%%%%%%
\topic{S \large\hskip -1ex ELECTED TALKS}{\small \\{\textcolor{gray}{\it $**$= invited talk}}}

%\smalltopic{Astronomy}{}
  \dateonly[**October 2016]{
      {\it Particle Size `by Proxy': Decoding the Textural Information in Remotely Sensed Landforms}\\
      School of Earth Sciences \& Environmental Sustainability, Northern Arizona University, Flagstaff, AZ.
  }

  \dateonly[July 2016]{
      {\it Stochasticity of riverbed backscattering, with implications for acoustical classification of non-cohesive sediment using multibeam sonar}\\
    8th International Conference on Fluvial Hydraulics, St. Louis, MO
  }

  \dateonly[**January 2016]{
      {\it The Digital Grain Size Web Computing Application}\\
      USGS Center for Data Integration, Denver, CO
  }

  \dateonly[**January 2016]{
      {\it Observations of sand dune migration on the Colorado River in Grand Canyon}\\
      Glen Canyon Dam Adaptive Management Program Adaptive Management Work Group Meeting, Phoenix, AZ
  }

  \dateonly[April 2015]{
      {\it Considerations for unsupervised riverbed sediment characterization using low-cost sidescan sonar: Examples from the Colorado River, AZ and the Penobscot River, ME.}\\
      Proceedings of the 10th Federal Interagency Sedimentation Conference, Reno, NV
  }

  \dateonly[April 2015]{
      {\it Using oblique digital photography for alluvial sandbar monitoring and low-cost change detection.}\\
      Proceedings of the 10th Federal Interagency Sedimentation Conference, Reno, NV
  }

  \dateonly[April 2015]{
      {\it Hydroacoustic signatures of Colorado riverbed sediments in Marble and Grand Canyons using multibeam sonar}\\
      Proceedings of the 10th Federal Interagency Sedimentation Conference, Reno, NV
  }

  \dateonly[March 2015]{
      {\it Acoustic and topographic sediment classification in Lower Marble Canyon}\\
      2nd MBES in Rivers Workshop, USGS Flagstaff, AZ
  }

  \dateonly[March 2015]{
      {\it Characterizing sand dune migration on the Colorado River in Western Grand Canyon using repeat multibeam mapping}\\
      2nd MBES in Rivers Workshop, USGS Flagstaff, AZ
  }

  \dateonly[March 2015]{
      {\it Towards automated substrate mapping with low-cost sidescan sonar}\\
      2nd MBES in Rivers Workshop, USGS Flagstaff, AZ
  }

  \dateonly[**February 2015]{
      {\it The Digital Grain Size Project: Past, Present and Future}\\
      USGS Coastal and Marine Geology, Woods Hole, MA
  }

  \dateonly[December 2014]{
      {\it Topographic and acoustic estimates of grain-scale roughness from high-resolution multibeam echo-sounder: examples from the Colorado River in Marble and Grand Canyons}\\
      American Geophysical Union Fall Meeting, San Francisco, CA
  }

  \dateonly[**August 2014]{
      {\it Measuring bed sediments for improved sediment budgets and physical habitat assessment}\\
      Glen Canyon Dam Adaptive Management Program Adaptive Management Work Group Meeting, Flagstaff, AZ
  }

  \dateonly[**February 2014]{
      {\it Bed Sediment Classification Using High-Frequency Acoustic Backscatter}\\
      Multibeam in Rivers Summit, Utah State University, Logan, UT
  }

  \dateonly[December 2013]{
      {\it Acoustic Scattering by an Heterogeneous River Bed: Relationship to Bathymetry and Implications for Sediment Classification using Multibeam Echosounder Data}\\
      American Geophysical Union Fall Meeting, San Francisco, CA
  }

  \dateonly[July 2012]{
      {\it Schmidt number of sand suspensions under oscillating-grid turbulence}\\
      33rd International Conference on Coastal Engineering, Santander, Spain
  }

  \dateonly[**July 2012]{
      {\it Digital Grain Size}\\
      British Geological Survey, Marine Geosciences Division, Edinburgh, UK
  }

  \dateonly[**February 2012]{
      {\it Nearshore Sediment Transport Through the Looking Glass}\\
      Grand Canyon Monitoring and Research Center, Flagstaff, AZ
  }

  \dateonly[February 2012]{
      {\it Co-variation of intertidal morphology, bedforms and grain size on a macrotidal sand beach: Praa Sands, UK}\\
      Ocean Sciences 2012, Salt Lake City, UT
  }

  \dateonly[December 2011]{
      {\it How do you tell how big something is without direct measurement? Estimating grain size using an image's spectrum}\\
      American Geophysical Union Fall Meeting, San Francisco, CA
  }

  \dateonly[November 2010]{
      {\it Hourly Measurements of Grain-Size from the Inner Continental Shelf Seabed Using a Fully-Automated, Hydraulically-Controlled Underwater Video Microscope}\\
      Particles in Europe 2010, Villefranche-Sur-Mer, France
  }

  \dateonly[June 2010]{
      {\it An automated and 'universal' method for measuring mean grain size from a digital image of sediment}\\
      9th Federal Interagency Sedimentation Conference, Las Vegas, NV
  }

  \dateonly[February 2010]{
      {\it Modeling sand resuspension and stratification in turbulent nearshore flows: sensitivity to grain size distribution}\\
      Ocean Sciences 2010, Portland, OR
    }

  \dateonly[**February 2010]{
      {\it Turbulence, Sediment Stratification and Altered Resuspension under Waves}\\
      Centre for Coastal Science and Engineering, University of Plymouth, UK
    }

  \dateonly[**January 2009]{
    {\it Morphodynamics and sediment dynamics of a macrotidal gravel beach}\\
    Coastal and Marine Geology, United States Geological Survey, Santa Cruz, CA
  }

  \dateonly[**October 2008]{
    {\it Optical sensing of gravel sediment transport and characteristics: recent advances and future challenges.}\\
    Lancaster University Environmental Imaging Network, Lancaster University, UK
  }

  \dateonly[September 2008]{
      {\it Granular Properties from Digital Images of Sediment: Implications for Coastal Sediment Transport Modelling}\\
      31st International Conference on Coastal Engineering (ICCE), Hamburg, Germany
  }

  \dateonly[December 2007]{
      {\it The relationship between sediment properties and sedimentation patterns on a macrotidal gravel beach over a semi lunar tidal cycle}\\
      American Geophysical Union Fall Meeting, San Francisco, CA
  }

  \dateonly[**November 2007]{
    {\it A year in the life of Slapton Sands - but was it a typical year?}\\
    Slapton Research Seminar, Field Studies Council, Slapton Ley, UK
  }

  \dateonly[May 2007]{
      {\it Field observations of step dynamics on a macrotidal gravel beach}\\
      Coastal Sediments 2007, New Orleans, LA
  }

  \dateonly[**December 2006]{
    {\it Field observations of morphological change and sediment dynamics from the nearshore of a gravel beach}\\
    Centre for Coastal Dynamics and Engineering (C-CoDE), University of Plymouth
  }

  \dateonly[**November 2006]{
    {\it A view from the beach}\\
    Slapton Research Seminar, Field Studies Council, Slapton Ley, UK
  }

  \dateonly[**December 2004]{
    {\it A tale of two storms}\\
    Slapton Research Seminar, Field Studies Council, Slapton Ley, UK
  }

\newpage

%%%%%%%%%%%%%%%%%%%%%%%%%%%%%%%%%%%%%%%%%%%%%%%%%%%%%%%%%%%%%%%%%%%%%%%%%%%
\topic{G \large\hskip -1ex RANTS}{~}


\subtopic[$\$$200,000]{USGS Mendenhall post-doctoral fellowship}{
Co-investigator: T. Sankey (PI), P. Grams, A. East, D. Buscombe., T. Sankey, (2015 – 2017). The fluvial-aeolian- hillslope continuum: measurement and modeling of topography and vegetation to inform landscape-scale connectivity for sediment in river valley ecosystems
}

\subtopic[$\$$46,417]{USGS Center for Data Integration}{
Principal-Investigator (2015 - 2016). The digital grain size web and mobile computing application
}

\subtopic[$\$$48,994]{USGS Innovation Fund}{
Principal-Investigator (2015 - 2016). LOBOS (Limnological and Oceanographic Benthic Observation System): The next generation dual-scale submersible benthic imaging system. Jointly funded by the USGS Innovation Fund ($\$$16,497), the Innovation Center for Earth Science Director's Fund ($\$$17,497) and the USGS Southwest Biological Science Center ($\$$15,000) 
}

\subtopic[$\$$4,253,400]{Glen Canyon Dam Adaptive Management Work Group}{
Co-Investigator (multiple PIs – J. Schmidt and others), (2015 - 2017). Sandbars and sediment storage dynamics: long-term monitoring and research at the site, research and ecosystem scales. Grand Canyon Monitoring and Research Center Triennial Work Plan
}

\subtopic[$\$$232,016]{National Park Service}{
Co-Investigator (multiple PIs – P.E. Grams and others), (2014 - 2017). Geomorphic Processes and Relations Among Flow Regime, Sediment Flux and Resource Conditions on the Green River in Canyonlands National Park
}

\subtopic[$\$$2,911,400]{Glen Canyon Dam Adaptive Management Work Group}{
Co-Investigator (multiple PIs – J. Schmidt and others), (2013 - 2014). Sandbars and sediment storage dynamics: long-term monitoring and research at the site, research and ecosystem scales. Grand Canyon Monitoring and Research Center Biennial Work Plan
}

\subtopic[$\pounds$240,000]{Engineering and Physical Sciences Research Council, UK}{
Co-Investigator; G. Masselink (PI), D.C. Conley, D. Buscombe., (2012 - 2014). Proto-type Experiment and Numerical Modelling of Energetic Sediment Transport under Waves (PESTS). EPSRC EP/K000306/1.
}

\subtopic[$\pounds$250]{Plymouth Marine Science Education Fund}{
Principal-Investigator (2008). Travel grant to attend and present at ICCE Hamburg 2008
}

\subtopic[$\pounds$150]{Challenger Society for Marine Science}{
Principal-Investigator (2008). Travel grant to attend and present at ICCE Hamburg 2008
}

\subtopic[$\$$500]{Society for Sedimentary Geology Grant}{
Principal-Investigator (2008). President's Fund to investigate nearshore bedload transport and bedforms with stereo underwater video cameras
}

\subtopic[$\$$2000]{International Association for Mathematical Geology}{
Principal-Investigator (2008). Grant to develop and trial algorithms for quantification of granular properties and coarse-grain sediment transport from images of the sea bed
}

\subtopic[700 euros]{International Association of Sedimentologists}{
Principal-Investigator (2008). Grant to investigate nearshore bedload transport and bedforms with stereo underwater video cameras
}

\subtopic[$\$$600]{American Geophysical Union}{
Principal-Investigator (2007). Travel Grant, to attend the AGU 2007 Fall Meeting in San Francisco, CA
}

\subtopic[$\pounds$300]{British Geomorphological Society}{
Principal-Investigator (2007). Postgraduate award, to attend and present at Coastal Sediments 2007, in New Orleans, LA
}

\newpage

%%%%%%%%%%%%%%%%%%%%%%%%%%%%%%%%%%%%%%%%%%%%%%%%%%%%%%%%%%%%%%%%%%%%%%%%%%%
\topic{P \large\hskip -1ex UBLICATIONS}{~}

\newcounter{PubNumber}
\newcommand{\publication}{\item[{\bf \textcolor{myblue}{[\stepcounter{PubNumber}\arabic{PubNumber}]}}]}


\subtopic{\hspace*{-3ex} 2016}{~
  %\vspace*{\baselineskip}
  \begin{itemize}[leftmargin=0ex, itemsep=0ex, parsep=.5ex, labelindent=-4ex]

    \publication
      D. Buscombe.
      {\sl Shallow water benthic imaging and substrate characterization using recreational-grade sidescan-sonar.}
      ENVIRONMENTAL MODELLING \& SOFTWARE, accepted 2016

    \publication
      M. Cuttler {\sl et al.}
      {\sl Estimating the settling velocity of bioclastic sediment from common grain-size analysis techniques}.
      SEDIMENTOLOGY, 10.1111/sed.12338, 2016

    \publication
      D. Hamill {\sl et al.}
      {\sl Towards bed texture change detection in large rivers from repeat imaging using recreational grade sidescan sonar}.
      Proceedings of the 8th International Conference on Fluvial Hydraulics, St. Louis, Missouri, July 2016.

    \publication
      D. Buscombe \& P.E. Grams.
      {\sl Stochasticity of riverbed backscattering, with implications for acoustical classification of non-cohesive sediment using multibeam sonar}.
      Proceedings of the 8th International Conference on Fluvial Hydraulics, St. Louis, Missouri, July 2016.

    \publication
      D. Buscombe.
      {\sl Spatially explicit spectral analysis of point clouds and geospatial data}.
      COMPUTERS \& GEOSCIENCES 86:92--108, 2016.

    \end{itemize}
}


\subtopic{\hspace*{-3ex} 2015}{~
  %\vspace*{\baselineskip}
  \begin{itemize}[leftmargin=0ex, itemsep=0ex, parsep=.5ex, labelindent=-4ex]

    \publication
      D. Buscombe {\sl et al.}
      {\sl Automated riverbed sediment classification using low-cost sidescan sonar}.
      JOURNAL OF HYDRAULIC ENGINEERING, 10.1061/(ASCE)HY.1943-7900.0001079, 06015019, 2015.

    \publication
      E.J. Davies {\sl et al.}
      {\sl A Evaluating Unsupervised Methods to Size and Classify Suspended Particles using Digital in-line Holography}.
      JOURNAL OF ATMOSPHERIC \& OCEANOGRAPHIC TECHNOLOGY 32:1241--1256, 2015.

    \publication
      P.E. Grams {\sl et al.}
      {\sl Use of Flux and Morphologic Sediment Budgets for Sandbar Monitoring on the Colorado River in Marble Canyon, Arizona}.
      Proceedings of the 10th Federal Interagency Sedimentation Conference, Reno, April, 2015.

    \publication
      D. Buscombe {\sl et al.}
      {\sl Hydroacoustic signatures of Colorado riverbed sediments in Marble and Grand Canyons using multibeam sonar}.
      Proceedings of the 10th Federal Interagency Sedimentation Conference, Reno, April, 2015.

    \publication
      D. Buscombe {\sl et al.}
      {\sl Considerations for unsupervised riverbed sediment characterization using low-cost sidescan sonar: Examples from the Colorado River, AZ and the Penobscot River, ME}.
      Proceedings of the 10th Federal Interagency Sedimentation Conference, Reno, April, 2015.

    \publication
      D. Buscombe {\sl et al.}
      {\sl Using oblique digital photography for alluvial sandbar monitoring and low-cost change detection}.
      Proceedings of the 10th Federal Interagency Sedimentation Conference, Reno, April, 2015.

    \end{itemize}
}

\subtopic{\hspace*{-3ex} 2014}{~
  %\vspace*{\baselineskip}
  \begin{itemize}[leftmargin=0ex, itemsep=0ex, parsep=.5ex, labelindent=-4ex]

    \publication
      J.A. Puleo {\sl et al.}
      {\sl A Comprehensive Field Study of Swash-Zone Processes, Part 1: Experimental Design with Examples of Hydrodynamic and Sediment Transport Measurements}.
      JOURNAL OF WATERWAY, PORT, COASTAL, \& OCEAN ENGINEERING 140:29–42, 2014.

    \publication
      D. Buscombe {\sl et al.}
      {\sl Autonomous bed-sediment imaging-systems for revealing temporal variability of grain size}.
      LIMNOLOGY \& OCEANOGRAPHY: METHODS 12:390 - 406, 2014.

    \publication
      D. Buscombe {\sl et al.}
      {\sl Characterizing riverbed sediment using high-frequency acoustics 1: Spectral properties of scattering}.
      JOURNAL OF GEOPHYSICAL RESEARCH - EARTH SURFACE 119:F003189, 2014.

    \publication
      D. Buscombe {\sl et al.}
      {\sl Characterizing riverbed sediment using high-frequency acoustics 2: Scattering signatures of Colorado River bed sediment in Marble and Grand Canyons}.
      JOURNAL OF GEOPHYSICAL RESEARCH - EARTH SURFACE 119:F003191, 2014.

    \end{itemize}
}

\subtopic{\hspace*{-3ex} 2013}{~
  %\vspace*{\baselineskip}
  \begin{itemize}[leftmargin=0ex, itemsep=0ex, parsep=.5ex, labelindent=-4ex]

    \publication
      D. Buscombe.
      {\sl Transferable Wavelet Method for Grain Size-Distribution from Images of Sediment Surfaces and Thin Sections, and Other Natural Granular Patterns}.
      SEDIMENTOLOGY 60:1709--1732, 2013.

    \end{itemize}
}

\subtopic{\hspace*{-3ex} 2012}{~
  %\vspace*{\baselineskip}
  \begin{itemize}[leftmargin=0ex, itemsep=0ex, parsep=.5ex, labelindent=-4ex]

    \publication
      J. Williams {\sl et al.}
      {\sl Barrier Dynamics Experiment (BARDEX): Aims, Design and Procedures}.
      COASTAL ENGINEERING 63:3-12, 2012.

    \publication
      D. Buscombe \& D. Conley
      {\sl Effective Shear Stress of Graded Sediment}.
      WATER RESOURCES RESEARCH, 48:W05506, 2012.

    \publication
      D. Buscombe \& D.M. Rubin.
      {\sl Advances in the Simulation and Automated Measurement of Granular Material, Part 1: Simulations}.
      JOURNAL OF GEOPHYSICAL RESEARCH - EARTH SURFACE 117:F02001, 2012.

    \publication
      D. Buscombe \& D.M. Rubin.
      {\sl Advances in the Simulation and Automated Measurement of Granular Material, Part 2: Direct Measures of Particle Properties}.
      JOURNAL OF GEOPHYSICAL RESEARCH - EARTH SURFACE 117:F02002, 2012.

    \publication
      J.R. Lacy {\sl et al.}
      {\sl Currents and sediment transport induced by a tsunami far from its source}.
      JOURNAL OF GEOPHYSICAL RESEARCH - OCEANS 117:C09028, 2012.

    \publication
      J. A. Puleo {\sl et al.}
      {\sl Comprehensive study of swash-zone hydrodynamics and sediment transport}.
      Proceedings of the 33rd International Conference on Coastal Engineering, Santander, July 2012.

    \publication
      D. Buscombe \& D. Conley
      {\sl Schmidt number of sand suspensions under oscillating-grid turbulence}.
      Proceedings of the 33rd International Conference on Coastal Engineering, Santander, July 2012.

    \publication
      D. Conley {\sl et al.}
      {\sl Use of digital holographic cameras to examine the measurement and understanding of sediment suspension in the nearshore}.
      Proceedings of the 33rd International Conference on Coastal Engineering, Santander, July 2012.

    \end{itemize}
}

\subtopic{\hspace*{-3ex} 2010}{~
  %\vspace*{\baselineskip}
  \begin{itemize}[leftmargin=0ex, itemsep=0ex, parsep=.5ex, labelindent=-4ex]

    \publication
      D. Buscombe {\sl et al.}
      {\sl Universal Approximation of Grain Size from Images of Non-Cohesive Sediment}.
      JOURNAL OF GEOPHYSICAL RESEARCH - EARTH SURFACE 115:F02015, 2010.

    \publication
      D. Buscombe {\sl et al.}
      {\sl An automated and 'universal' method for measuring mean grain size from a digital image of sediment}.
      Proceedings of the 9th Federal Interagency Sedimentation Conference, Las Vegas, June 2010.

    \end{itemize}
}

\subtopic{\hspace*{-3ex} 2009}{~
  %\vspace*{\baselineskip}
  \begin{itemize}[leftmargin=0ex, itemsep=0ex, parsep=.5ex, labelindent=-4ex]

    \publication
      D. Buscombe \& G. Masselink.
      {\sl Grain Size Information from the Statistical Properties of Digital Images of Sediment}.
      SEDIMENTOLOGY 56:421-438, 2009.

    \publication
      J.A. Warrick {\sl et al.}
      {\sl Cobble Cam: Grain-size measurements of sand to boulder from digital photographs and autocorrelation analyses}.
      EARTH SURFACE PROCESSES \& LANDFORMS 34:1811-1821, 2009.

    \publication
      J. Williams {\sl et al.}
      {\sl BARDEX (Barrier Dynamics Experiment): taking the beach into the laboratory}.
      JOURNAL OF COASTAL RESEARCH SI 56:158-162, 2009.

    \end{itemize}
}

\subtopic{\hspace*{-3ex} 2008}{~
  %\vspace*{\baselineskip}
  \begin{itemize}[leftmargin=0ex, itemsep=0ex, parsep=.5ex, labelindent=-4ex]

    \publication
      G. Masselink {\sl et al.}
      {\sl Sediment Trend Models Fail to Reproduce Small Scale Sediment Transport Patterns on an Intertidal Beach}.
      SEDIMENTOLOGY 55:667-687, 2008.

    \publication
      M. Austin \& D. Buscombe.
      {\sl Morphological Change and Sediment Dynamics of the Beach Step on a Macrotidal Gravel Beach}.
      MARINE GEOLOGY 249:167-183, 2008.

    \publication
      D. Buscombe.
      {\sl Estimation of Grain Size Distributions and Associated Parameters from Digital Images of Sediment}.
      SEDIMENTARY GEOLOGY 210:1-10, 2008.

    \publication
      G. Masselink \& D. Buscombe.
      {\sl Shifting gravel: A case study of Slapton Sands}.
      GEOGRAPHY REVIEW 22:27-31, 2008.

    \publication
      D. Buscombe {\sl et al.}
      {\sl Granular Properties from Digital Images of Sediment: Implications for Coastal Sediment Transport Modelling}.
      Proceedings of the 31st International Conference on Coastal Engineering (ICCE), Hamburg, 2008.

    \publication
      A. Ruiz de Alegria {\it et al.}
      {\it Storm Impacts on a Gravel Beach Using the ARGUS video system}.
      Proceedings of the 31st International Conference on Coastal Engineering (ICCE), Hamburg, 2008.

    \publication
      M. Austin {\sl et al.}
      {\it Groundwater seepage between a gravel barrier beach and a freshwater lagoon}.
      Proceedings of the 31st International Conference on Coastal Engineering (ICCE), Hamburg, 2008.

    \end{itemize}
}

\subtopic{\hspace*{-3ex} 2006--2007}{~
  %\vspace*{\baselineskip}
  \begin{itemize}[leftmargin=0ex, itemsep=0ex, parsep=.5ex, labelindent=-4ex]

    \publication
      D. Buscombe {\sl et al.}
      {\it Field observations of step dynamics on a macrotidal gravel beach}.
      In Kraus, N., and Rosati, J., (Eds), Proceedings of Coastal Sediments 2007 (Volume 1), 2007.

    \publication
      D. Buscombe \& G. Masselink.
      {\it Concepts in Gravel Beach Dynamics}.
      EARTH SCIENCE REVIEWS 79:33-52, 2006.

  \end{itemize}

}

\end{document}


        \item[36] Hamill, D., {\bf Buscombe, D.}, Wheaton, J.M., Melis, T.S., Grams. P.E., 2016, Towards bed texture change detection in large rivers from repeat imaging using recreational grade sidescan sonar {\sl Proceedings of the 8th International Conference on Fluvial Hydraulics}, St. Louis, Missouri, July 2016.\\
        
        \item[35] {\bf Buscombe, D.}, Grams. P.E., 2016, Stochasticity of riverbed backscattering, with implications for acoustical classification of non-cohesive sediment using multibeam sonar {\sl Proceedings of the 8th International Conference on Fluvial Hydraulics}, St. Louis, Missouri, July 2016.\\

	\item[34] {\bf 	Buscombe, D.}, 2016, Spatially explicit spectral analysis of point clouds and geospatial data. {\sl COMPUTERS \& GEOSCIENCES} 86, 92-108, 10.1016/j.cageo.2015.10.004.

	\end{list1}

	\subsection{\mysidestyle 2015}
	\begin{list1}
		
	\item[33] {\bf Buscombe, D.}, Grams, P.E., Smith, S.M., 2015, Automated riverbed sediment classification using low-cost sidescan sonar. {\sl JOURNAL OF HYDRAULIC ENGINEERING}, 10.1061/(ASCE)HY.1943-7900.0001079, 06015019.\\
	
	\item[32] Davies, E.J., {\bf Buscombe, D.}, Graham, G.W., Nimmo-Smith, W.A.M., 2015, Evaluating Unsupervised Methods to Size and Classify Suspended Particles using Digital in-line Holography, {\sl JOURNAL OF ATMOSPHERIC \& OCEANOGRAPHIC TECHNOLOGY}, 32, 1241 - 1256. doi: 10.1175/JTECH-D-14-00157.1 \\

        \item[31] Grams. P.E., {\bf Buscombe, D.}, Topping, D.J., Hazel, J.E., and Kaplinski, M.A. (2015) Use of Flux and Morphologic Sediment Budgets for Sandbar Monitoring on the Colorado River in Marble Canyon, Arizona. {\sl Proceedings of the 10th Federal Interagency Sedimentation Conference}, Reno, April 2015.\\

        \item[30] {\bf Buscombe, D.}, Grams. P.E., Kaplinski, M.A., Tusso, R.B., and Rubin, D.M. (2015) Hydroacoustic signatures of Colorado riverbed sediments in Marble and Grand Canyons using multibeam sonar. {\sl Proceedings of the 10th Federal Interagency Sedimentation Conference}, Reno, April 2015.\\

        \item[29] {\bf Buscombe, D.}, Grams. P.E., Melis, T.S., Smith, S. (2015) Considerations for unsupervised riverbed sediment characterization using low-cost sidescan sonar: Examples from the Colorado River, AZ and the Penobscot River, ME. {\sl Proceedings of the 10th Federal Interagency Sedimentation Conference}, Reno, April 2015.\\

        \item[28] {\bf Buscombe, D.}, Tusso, R.B., Grams. P.E. (2015) Using oblique digital photography for alluvial sandbar monitoring and low-cost change detection. {\sl Proceedings of the 10th Federal Interagency Sedimentation Conference}, Reno, April 2015.
	
	\end{list1}
	
	\subsection{\mysidestyle 2014}
	\begin{list1}
        \item[27] Puleo, J., Blenkinsopp, C., Conley, D., Masselink, G., Turner, I., Russell, P., {\bf Buscombe, D.}, Howe, D., Lanckriet, T., McCall, R., and Poate, T., 2014, A Comprehensive Field Study of Swash-Zone Processes, Part 1: Experimental Design with Examples of Hydrodynamic and Sediment Transport Measurements. {\sl JOURNAL OF WATERWAY, PORT, COASTAL, \& OCEAN ENGINEERING}, 140, 29–42. 10.1061/(ASCE)WW.1943-5460.0000210.\\

	\item[26]  {\bf Buscombe, D.}, Rubin, D.M., Lacy, J.R., Storlazzi, C., Hatcher, G., Chezar, H., Wyland, R. and Sherwood, C., 2014, Autonomous bed-sediment imaging-systems for revealing temporal variability of grain size. {\sl LIMNOLOGY \& OCEANOGRAPHY: METHODS}, 12, 390 - 406. \\

	\item[25] {\bf Buscombe, D.}, Grams, P.E., Kaplinski, M.A., 2014, Characterizing riverbed sediment using high-frequency acoustics 1: Spectral properties of scattering. {\sl JOURNAL OF GEOPHYSICAL RESEARCH - EARTH SURFACE, 119, doi:10.1002/2014JF003189.}.\\

	\item[24] {\bf Buscombe, D.}, Grams, P.E., Kaplinski, M.A., 2014, Characterizing riverbed sediment using high-frequency acoustics 2: Scattering signatures of Colorado River bed sediment in Marble and Grand Canyons. {\sl JOURNAL OF GEOPHYSICAL RESEARCH - EARTH SURFACE, 119, doi:10.1002/2014JF003191}.

	\end{list1}

	\subsection{\mysidestyle 2013}
	\begin{list1}
        \item[23] {\bf Buscombe, D.}, 2013, Transferable Wavelet Method for Grain Size-Distribution from Images of Sediment Surfaces and Thin Sections, and Other Natural Granular Patterns. {\sl SEDIMENTOLOGY} 60, 1709--1732. 

	\end{list1}

	\subsection{\mysidestyle 2012}
	\begin{list1}
	\item[22] Williams, J.J., {\bf Buscombe, D.}, Masselink, G., Turner, I., and Swinkels, C., 2012, Barrier Dynamics Experiment (BARDEX): Aims, Design and Procedures. {\sl COASTAL ENGINEERING} 63, 3-12.\\

	\item[21] {\bf Buscombe, D.}, and Conley, D.C., 2012, Effective Shear Stress of Graded Sediment. {\sl WATER RESOURCES RESEARCH} 48, W05506.\\

	\item[20] {\bf Buscombe, D.}, and Rubin, D.M., 2012, Advances in the Simulation and Automated Measurement of Granular Material, Part 1: Simulations. {\sl JOURNAL OF GEOPHYSICAL RESEARCH - EARTH SURFACE} 117, F02001.\\

	\item[19] {\bf Buscombe, D.}, and Rubin, D.M., 2012, Advances in the Simulation and Automated Measurement of Granular Material, Part 2: Direct Measures of Particle Properties. {\sl JOURNAL OF GEOPHYSICAL RESEARCH - EARTH SURFACE} 117, F02002.\\

	\item[18] Lacy, J.R., Rubin, D.M. and {\bf Buscombe, D.}, 2012, Currents and sediment transport induced by a tsunami far from its source. {\sl JOURNAL OF GEOPHYSICAL RESEARCH - OCEANS} 117, C09028.\\

	\item[17] Puleo, J.A., Conley, D.C., Masselink, G., Russell, P., Turner, I.L., Blenkinsopp, C., {\bf Buscombe, D.}, Lanckriet, T., McCall, R., and Poate, T. (2012) Comprehensive study of swash-zone hydrodynamics and sediment transport. {\sl Proceedings of the 33rd International Conference on Coastal Engineering}, Santander, July 2012.\\

	\item[16] {\bf Buscombe, D.}, and Conley, D.C. (2012) Schmidt number of sand suspensions under oscillating-grid turbulence. {\sl Proceedings of the 33rd International Conference on Coastal Engineering}, Santander, July 2012.\\

	\item[15] Conley, D.C., {\bf Buscombe, D.}, and Nimmo-Smith, A. (2012) Use of digital holographic cameras to examine the measurement and understanding of sediment suspension in the nearshore. {\sl Proceedings of the 33rd International Conference on Coastal Engineering}, Santander, July 2012.

	\end{list1}

	\subsection{\mysidestyle 2010}
	\begin{list1}
	\item[14] {\bf Buscombe, D.}, Rubin, D.M., and Warrick, J.A., 2010, Universal Approximation of Grain Size from Images of Non-Cohesive Sediment. {\sl JOURNAL OF GEOPHYSICAL RESEARCH - EARTH SURFACE} 115, F02015.\\

	\item[13] {\bf Buscombe, D.}, Rubin, D. M., and Warrick, J. A. (2010) An automated and 'universal' method for measuring mean grain size from a digital image of sediment. {\sl Proceedings of the 9th Federal Interagency Sedimentation Conference}, Las Vegas June 2010.

	\end{list1}
	
	\subsection{\mysidestyle 2009}
	\begin{list1}
	\item[12] {\bf Buscombe, D.}, and Masselink, G., 2009, Grain Size Information from the Statistical Properties of Digital Images of Sediment. {\sl SEDIMENTOLOGY} 56, 421-438 \\

	\item[11] Warrick, J.A., Rubin, D.M., Ruggiero, P., Harney, J., Draut, A.E., and {\bf Buscombe, D.}, 2009, Cobble Cam: Grain-size measurements of sand to boulder from digital photographs and autocorrelation analyses. {\sl EARTH SURFACE PROCESSES \& LANDFORMS} 34, 1811-1821.\\

	\item[10] Williams, J., Masselink, G., {\bf Buscombe, D.}, Turner, I., Matias, A., Ferreira, O., Meltje, N., Bradbury, A., Albers, T., and Pan, S., 2009, BARDEX (Barrier Dynamics Experiment): taking the beach into the laboratory. {\sl JOURNAL OF COASTAL RESEARCH} SI 56, 158-162.
















