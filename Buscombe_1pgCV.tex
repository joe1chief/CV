
%________________________________________________________________________________________
% @brief    LaTeX2e Resume for Daniel Buscombe
% @author   Daniel Buscombe
% @date     June 2013

%________________________________________________________________________________________
\documentclass[margin,line]{resume}

\usepackage{url}

\begin{document}
\name{\Large Daniel Buscombe}
\begin{resume}

Research Geologist, \\
United States Geological Survey,\\
Grand Canyon Monitoring and Research Center. \\
http://dbuscombe-usgs.github.com

%    %____________________________________________________________________________________
    % Contact Information
    \section{\mysidestyle Contact\\Information}\vspace{2mm}


    \begin{tabular}{@{} l @{\hspace{20mm}} r}
    USGS, Grand Canyon Monitoring and Research Center & Tel: 928-726-7216  \\
    2255 N Gemini Drive, Flagstaff, AZ 86001 & email: dbuscombe@usgs.gov \\
    \end{tabular}

    %____________________________________________________________________________________
%    % Research Interests
%\section{\mysidestyle Research\\Interests}
%     I am a geomorphologist/sedimentologist with a principal research interest in sediment dynamics, which includes the physical and biochemical makeup of sediment beds in a range of aquatic environments, the role of sediment heterogeneity in the physics of sediment transport, and the morphodynamics (both micro- and macro-scale) that result from sediment transport, sedimentation, stratigraphy and geomorphological forms. I study the complex inter-relations between fluid flows, geomorphology, sediment transport and sedimentology. I investigate these processes by developing novel field-deployed optical and acoustic imaging systems, and computational algorithms for small-scale sediment hydroacoustics, in-situ particle and bed imaging and flow-field/turbulence measurements. 


    %____________________________________________________________________________________
    % Education
    \section{\mysidestyle Education}

        \begin{footnotesize}
    {\bf Ph.D. (2008), Coastal Geomorphology/Nearshore Oceanography, University of Plymouth}, Plymouth, UK. {\sl Morphodynamics, Sediment Dynamics and Sedimentation of a Gravel Beach}. Advisor:  Prof. Gerhard Masselink.\vspace{2mm}

    {\bf BSc (Hons), 1st class (2003), Physical Geography with Minors in Environemntal Sciences and Biology, Lancaster University}, Lancaster, UK.  {\sl Morphodynamics of a Ridge-and-Runnel System on a Macrotidal Beach}. Advisor:  Dr Suzanna Ilic. \vspace{2mm}
        \end{footnotesize}

    %____________________________________________________________________________________
    % Professional Experience
    \section{\mysidestyle Employment\\History}

        \begin{footnotesize}
    {\bf November 2012 --  present}. {\sl Research Geologist, Grand Canyon Monitoring and Research Center, U.S. Geological Survey, Flagstaff, AZ, USA.}

    {\bf October 2009 -- November 2012}. {\sl Post-doctoral Research Fellow, School of Marine Science \& Engineering, University of Plymouth, UK.} 

    {\bf September, 2008 -- 2011}. {\sl Computer Programming Contractor, Marine Biology \& Ecology Research Centre, University of Plymouth, UK.} 

    {\bf October, 2008 -- October 2009}. {\sl Post-doctoral Research Scholar, United States Geological Survey, Santa Cruz, California, USA.} 

    {\bf June, 2008 -- September, 2008}. {\sl Research Assistant, School of Geography, University of Plymouth, UK.} 

    {\bf December, 2007 -- April, 2008}. {\sl Research Assistant, School of Earth, Ocean \& Environmental Science, University of Plymouth, UK.} 

    {\bf October, 2004 -- July 2008}. {\sl Associate Lecturer and Demonstrator (part-time), School of Geography, University of Plymouth, UK.}

    {\bf August 2003 - September, 2004}. {\sl Assistant tutor, Field Studies Council, Castle Head, Grange-over-Sands, UK}.
        \end{footnotesize}

    %____________________________________________________________________________________
    % Publications
    \section{\mysidestyle Ten Selected Publications}

        \begin{footnotesize}
	\begin{list1}
	\item[1] {\bf Buscombe, D.}, and Masselink, G. (2006) Concepts in Gravel Beach Dynamics. {\sl Earth Science Reviews} 79, 33-52.
	
	\item[2] Masselink, G., {\bf Buscombe, D.}, Austin, M.J, O'Hare, T., Russell, P. (2008) Sediment Trend Models Fail to Reproduce Small Scale Sediment Transport Patterns on an Intertidal Beach. {\sl Sedimentology} 55, 667-687.
	
	\item[3] Austin, M.J., and {\bf Buscombe, D.} (2008) Morphological Change and Sediment Dynamics of the Beach Step on a Macrotidal Gravel Beach. {\sl Marine Geology} 249, 167-183. 

	\item[4] {\bf Buscombe, D.}, Rubin, D.M., and Warrick, J.A. (2010) Universal Approximation of Grain Size from Images of Non-Cohesive Sediment. {\sl Journal of Geophysical Research - Earth Surface} 115, F02015.

	\item[5] Williams, J.J., {\bf Buscombe, D.}, Masselink, G., Turner, I., and Swinkels, C. (2012) Barrier Dynamics Experiment (BARDEX): Aims, Design and Procedures. {\sl Coastal Engineering} 63, 3-12.

	\item[6] {\bf Buscombe, D.}, and Conley, D.C. (2012) Effective Shear Stress of Graded Sediment. {\sl Water Resources Research} 48, W05506.

	\item[7] {\bf Buscombe, D.}, and Rubin, D.M. (2012) Advances in the Simulation and Automated Measurement of Granular Material, Part 1: Simulations. {\sl Journal of Geophysical Research - Earth Surface} 117, F02001.

	\item[8] {\bf Buscombe, D.}, and Rubin, D.M. (2012) Advances in the Simulation and Automated Measurement of Granular Material, Part 2: Direct Measures of Particle Properties. {\sl Journal of Geophysical Research - Earth Surface} 117, F02002.

	\item[9] Lacy, J.R., Rubin, D.M. and {\bf Buscombe, D.} (2012) Currents and sediment transport induced by a tsunami far from its source. {\sl Journal of Geophysical Research - Oceans} 117, C09028.

        \item[10] {\bf Buscombe, D.} (2013) Transferable Wavelet Method for Grain Size-Distribution from Images of Sediment Surfaces and Thin Sections, and Other Natural Granular Patterns. {\sl Sedimentology} 60, 1709--1732. DOI: 10.1111/sed.12049


	\end{list1}
        \end{footnotesize}
        

%________________________________________________________________________________________
\end{resume}
\end{document}

%________________________________________________________________________________________
% EOF
