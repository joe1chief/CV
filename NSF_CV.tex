\documentstyle[11pt,aaspp4]{article}

%% latex JVcv;  dvips -o JVcv.ps JVcv; ps2pdf JVcv.ps    

\begin{document}

\def\heading{\noindent}
\def\ref{\vskip -0.05in \hangindent 2.5pc \hangafter 1}
\def\listing{\vskip -0.1in \hangindent 2.5pc \hangafter 1}
\def\Bline#1{\hskip 0.3in $\bullet$ #1 \\ \smallskip}
\def\Sline#1{\hskip 0.6in #1 \\}
\def\line#1{#1 \\}


\centerline{\bf Biographical Sketch for Daniel Buscombe}

\heading {\bf  Professional Preparation}

\hskip -0.1in \begin{tabular}{llll}
  University of Plymouth &  Nearshore Oceanography        & Ph.D.   & 2004 - 2008 \\
  Lancaster University           &  Physical Geography & B.Sc.   & 2000 - 2003 \\
\end{tabular}

\heading {\bf  Appointments and Teaching Experience}

\hskip -0.1in \begin{tabular}{lll}
  U.S Geological Survey    &  Research Geologist & 2012 - Present \\
  University of Plymouth &  NERC Postdoctoral Research Fellow & 2009 - 2012\\
  UC Santa Cruz &  Postdoctoral Research Fellow & 2008 - 2009 \\
  University of Plymouth &  Teaching \& Research Assistant & 2004 - 2008\\
%  University of Plymouth &  Teaching Assistant & 2004 - 2008\\
  Field Studies Council & Assistant Tutor & 2003 - 2004\\
\end{tabular}

\heading {\bf Selected Related Publications} 

\ref
 {\bf D. Buscombe}, P. Grams, \& M. Kaplinski. {\it Characterizing riverbed sediment using high-frequency acoustics 2: Scattering signatures of Colorado River bed sediment in Marble and Grand Canyons}. Journal of Geophysical Research 119:doi:10.1002/2014JF003191, 2014.

\ref
 {\bf D. Buscombe}. {\it Transferable Wavelet Method for Grain Size-Distribution from Images of Sediment Surfaces and Thin Sections, and Other Natural Granular Patterns}. Sedimentology 60: 1709--1732, 2013.

\ref
 J. Lacy, D. Rubin, \& {\bf D. Buscombe}. {\it Currents, Drag and Sediment Transport Induced by a Tsunami}. Journal of Geophysical Research 117:C09028, 2012.

\ref
 M. Austin \& {\bf D. Buscombe}. {\it Morphological Change and Sediment Dynamics of the Beach Step on a Macrotidal Gravel Beach}. Marine Geology 249:167--183, 2008.

\ref
 {\bf D. Buscombe} \& G. Masselink. {\it Concepts in gravel beach dynamics}. Earth Science Reviews 79:33--52, 2006.

\heading {\bf Other Significant Publications}

\ref
{\bf D. Buscombe}, D. Rubin, J. Lacy, C. Storlazzi, G. Hatcher, H. Chezar, R. Wyland, \& C. Sherwood. {\it Autonomous bed-sediment imaging-systems for revealing temporal variability of grain size}. Limnology and Oceanography: Methods 12:390--406, 2014

\ref
J. Williams, {\bf D. Buscombe}, G. Masselink, I. Turner, \& C. Swinkels. {\it Barrier Dynamics Experiment (BARDEX): Aims, Design and Procedures}. Coastal Engineering 63:3--12, 2012.

\ref
{\bf D. Buscombe} \& D. Conley. {\it Effective Shear Stress of Graded Sediment}. Water Resources Research 48:W05506, 2012.

\ref
{\bf D. Buscombe}, D. Rubin, \& J. Warrick {\it Universal Approximation of Grain Size from Images of Non-Cohesive Sediment}. Journal of Geophysical Research 115:F02015, 2010.

\ref
G. Masselink, {\bf D. Buscombe}, M. Austin, T. O'Hare, \& P. Russell {\it Sediment Trend Models Fail to Reproduce Small Scale Sediment Transport Patterns on an Intertidal Beach}. Sedimentology 55:667--687, 2008.

\heading {\bf Synergistic Activities}

\listing
{\bf Open Source Contributions:} I'm leading several open-source community software development initiatives for processing data from sidescan sonar, multibeam sonar, photogrammetry, holographic imagery, sediment imagery, point clouds, and many more (https://github.com/dbuscombe-usgs), motivated by a passion for transparency and reproducibility in data analysis and research, using open-source tools.

\listing
{\bf Academic-Industry Partnership:} I'm part of the steering committee of an informal group of multibeam sonar users (from industry, academia, and other agencies) in shallow water environments (https://sites.google.com/site/mbesinriverssummitworkshop/).

\listing
{\bf Science Communication:} I have convened two sessions at Fall AGU Conferences: 1) in 2007, on ‘Linking sediment supply, bed-sediment particle size, sediment transport and bed morphology in fluvial, marine, and Aeolian settings’, and 2) in 2013: on ‘Fluvial sediment budgets: can we do better?’. I convened the Young Coastal Scientists and Engineers Conference, at Plymouth University in 2007. I also convened the Quaternary Research Association’s 4th International Postgraduate Symposium, at Plymouth University in 2005.

\listing
{\bf Teaching \& Mentoring:} I'm involved in casual teaching at NAU, and I’m currently training and advising two PhD and three Masters students in my research program. In 2016, I conceived, organized and ran a 3 day-long Software Carpentry workshop for 30 of my colleagues at the U.S. Geological Survey, teaching computing skills for scientists.

\listing
{\bf Scientific Review:} Frequent ad hoc reviewer for 12 journals, and 2 science research councils..


\heading {\bf Selected Collaborators}

 \hskip -0.1in \begin{tabular}{ll}
Paul Grams, {\it U.S. Geological Survey} & Jon Warrick, {\it U.S. Geological Survey}\\
David Rubin, {\it University of California Santa Cruz} & David Topping, {\it U.S. Geological Survey}\\
Daniel Conley, {\it University of Plymouth} & Joe Wheaton, {\it Utah State University}\\
Alex Nimmo-Smith, {\it University of Plymouth} & Jack Puleo, {\it University of Delaware}\\
Jessie Lacy, {\it U.S. Geological Survey} & Sean Smith, {\it University of Maine}\\
Emlyn Davies, {\it SINTEF, Norway} & Gerd Masselink, {\it University of Plymouth}\\
Martin Austin, {\it Bangor University} & Brandon McElroy, {\it University of Wyoming}\\
\end{tabular}

\heading {\bf Graduate Advisors}

\hskip -0.1in \begin{tabular}{lll}
  Gerd Masselink & {\it University of Plymouth} & (2004-2008)\\
  Mark Davidson & {\it University of Plymouth} & (2004-2008)\\
\end{tabular}

\end{document}
