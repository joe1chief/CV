\documentstyle[11pt,aaspp4]{article}

%% latex JVcv;  dvips -o JVcv.ps JVcv; ps2pdf JVcv.ps    

\begin{document}

\def\heading{\noindent}
\def\ref{\vskip -0.05in \hangindent 2.5pc \hangafter 1}
\def\listing{\vskip -0.1in \hangindent 2.5pc \hangafter 1}
\def\Bline#1{\hskip 0.3in $\bullet$ #1 \\ \smallskip}
\def\Sline#1{\hskip 0.6in #1 \\}
\def\line#1{#1 \\}


\centerline{\bf Biographical Sketch for Daniel Buscombe}

\heading {\bf  Research Interests}
\vskip -0.1in 
I study benthic sedimentary environments. Most of my research involves developing and applying novel acoustics and optics instrumentation and computational algorithms to remotely sense the properties of flows of water and sediment over landforms in rivers, lakes and coastal seas. My approach is interdisciplinary in sedimentology, coastal and hydraulic engineering, and geophysics, applying methodologies ranging from field surveys and laboratory analysis to analytical and numerical modeling.

\heading {\bf  Professional Preparation}

\hskip -0.1in \begin{tabular}{llll}
  University of Plymouth, UK &  Nearshore Oceanography        & Ph.D.   & 2004 - 2008 \\
  Lancaster University, UK           &  Physical Geography & B.Sc.   & 2000 - 2003 \\
\end{tabular}

\heading {\bf  Professional Appointments}

\hskip -0.1in \begin{tabular}{lll}
  Northern Arizona University, USA    &  Assistant Research Professor & 2016 - Present \\
  U.S Geological Survey, USA    &  Research Geologist & 2012 - 2016 \\
  University of Plymouth, UK &  NERC Postdoctoral Research Fellow & 2009 - 2012\\
  UC Santa Cruz, USA &  Postdoctoral Research Fellow & 2008 - 2009 \\
  University of Plymouth, UK &  Teaching \& Research Assistant & 2004 - 2008\\
%  University of Plymouth &  Teaching Assistant & 2004 - 2008\\
  Field Studies Council, UK & Assistant Tutor & 2003 - 2004\\
\end{tabular}

\heading {\bf Six Publications Related to This Proposal} 

\ref
 {\bf D. Buscombe}, P. Grams, \& M. Kaplinski. {\it Compositional signatures in acoustic backscatter over vegetated and unvegetated mixed sand-gravel riverbeds}. Journal of Geophysical Research 122:doi:10.1002/2017JF004302, 2017.

\ref
 {\bf D. Buscombe} {\it Shallow water benthic imaging and substrate characterization using recreational-grade sidescan-sonar}. Environmental Modelling \& Software 89, 1-18, 2017.

\ref
 {\bf D. Buscombe} {\it Spatially explicit spectral analysis of point clouds and geospatial data}. Computers \& Geosciences 86:92 - 108, 2016.

\ref
 {\bf D. Buscombe}, P. Grams, \& S. Smith. {\it Automated riverbed sediment classification using low-cost sidescan sonar}. Journal of Hydraulic Engineering: 06015019, 2016.

\ref
 {\bf D. Buscombe}, P. Grams, \& M. Kaplinski. {\it Characterizing riverbed sediment using high-frequency acoustics 2: Scattering signatures of Colorado River bed sediment in Marble and Grand Canyons}. Journal of Geophysical Research 119:doi:10.1002/2014JF003191, 2014.

\ref
J. Williams, {\bf D. Buscombe}, G. Masselink, I. Turner, \& C. Swinkels. {\it Barrier Dynamics Experiment (BARDEX): Aims, Design and Procedures}. Coastal Engineering 63:3--12, 2012.


\heading {\bf Six Other Significant Publications}

\ref
{\bf D. Buscombe}, D. Rubin, J. Lacy, C. Storlazzi, G. Hatcher, H. Chezar, R. Wyland, \& C. Sherwood. {\it Autonomous bed-sediment imaging-systems for revealing temporal variability of grain size}. Limnology and Oceanography: Methods 12:390--406, 2014

\ref
 {\bf D. Buscombe}. {\it Transferable Wavelet Method for Grain Size-Distribution from Images of Sediment Surfaces and Thin Sections, and Other Natural Granular Patterns}. Sedimentology 60: 1709--1732, 2013.

\ref
{\bf D. Buscombe} \& D. Conley. {\it Effective Shear Stress of Graded Sediment}. Water Resources Research 48:W05506, 2012.

\ref
 J. Lacy, D. Rubin, \& {\bf D. Buscombe}. {\it Currents, Drag and Sediment Transport Induced by a Tsunami}. Journal of Geophysical Research 117:C09028, 2012.

\ref
{\bf D. Buscombe}, D. Rubin, \& J. Warrick {\it Universal Approximation of Grain Size from Images of Non-Cohesive Sediment}. Journal of Geophysical Research 115:F02015, 2010.

\ref
 {\bf D. Buscombe} \& G. Masselink. {\it Concepts in gravel beach dynamics}. Earth Science Reviews 79:33--52, 2006.

%\ref
% M. Austin \& {\bf D. Buscombe}. {\it Morphological Change and Sediment Dynamics of the Beach Step on a Macrotidal Gravel Beach}. Marine Geology 249:167--183, 2008.

%\ref
%G. Masselink, {\bf D. Buscombe}, M. Austin, T. O'Hare, \& P. Russell {\it Sediment Trend Models Fail to Reproduce Small Scale Sediment Transport Patterns on an Intertidal Beach}. Sedimentology 55:667--687, 2008.

\heading {\bf Synergistic Activities}

\listing
{\bf Open Source Contributions:} I'm leading several open-source community software development initiatives for processing data from sidescan sonar, multibeam sonar, photogrammetry, holographic imagery, sediment imagery, point clouds, and many more (https://github.com/dbuscombe-usgs), motivated by a passion for transparency and reproducibility in data analysis and research, using open-source tools.

\listing
{\bf Academic-Industry Partnership:} I currently serve on the NSF-funded National Ecological Observatory Network Aquatic Technical Working Group, advising on bathymetry, substrate characterisation, and hydroacoustic instrumentation and analyses. I'm also part of the steering committee of an informal group of multibeam sonar users (from industry, academia, and other agencies) in shallow water environments (https://sites.google.com/site/mbesinriverssummitworkshop/).

\listing
{\bf Science Communication:} I have convened two sessions at Fall AGU Conferences: 1) in 2007, on ‘Linking sediment supply, bed-sediment particle size, sediment transport and bed morphology in fluvial, marine, and Aeolian settings’, and 2) in 2013: on ‘Fluvial sediment budgets: can we do better?’. I also convened the 2nd 'Multibeam in Rivers' Workshop at the U.S. Geological Survey in 2015, the 'Young Coastal Scientists and Engineers Conference' at Plymouth University in 2007, and the Quaternary Research Association’s 4th International Postgraduate Symposium at Plymouth University in 2005.

\listing
{\bf Teaching \& Mentoring:} I'm involved in formal and casual teaching at NAU, and I’m currently training and advising three graduate students (two PhD and one Masters) in my research program. In 2016, I conceived, organized and ran a 3 day-long Software Carpentry workshop for 30 of my colleagues at the U.S. Geological Survey, teaching computing skills for scientists.

\listing
{\bf Scientific Review:} Frequent ad hoc reviewer for 15 journals (in the fields of hydraulic and coastal engineering, sedimentology, and geomorphology), and 2 science research councils in the United States and United Kingdom.


%\heading {\bf Selected Collaborators}

% \hskip -0.1in \begin{tabular}{ll}
%Paul Grams, {\it U.S. Geological Survey} & Jon Warrick, {\it U.S. Geological Survey}\\
%David Rubin, {\it University of California Santa Cruz} & David Topping, {\it U.S. Geological Survey}\\
%Daniel Conley, {\it University of Plymouth} & Joe Wheaton, {\it Utah State University}\\
%Alex Nimmo-Smith, {\it University of Plymouth} & Jack Puleo, {\it University of Delaware}\\
%Jessie Lacy, {\it U.S. Geological Survey} & Sean Smith, {\it University of Maine}\\
%Emlyn Davies, {\it SINTEF, Norway} & Gerd Masselink, {\it University of Plymouth}\\
%Martin Austin, {\it Bangor University} & Brandon McElroy, {\it University of Wyoming}\\
%\end{tabular}

%\heading {\bf Graduate Advisors}

%\hskip -0.1in \begin{tabular}{lll}
%  Gerd Masselink & {\it University of Plymouth} & (2004-2008)\\
%  Mark Davidson & {\it University of Plymouth} & (2004-2008)\\
%\end{tabular}

\end{document}
