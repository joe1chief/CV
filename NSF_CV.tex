\documentstyle[11pt,aaspp4]{article}

%% latex JVcv;  dvips -o JVcv.ps JVcv; ps2pdf JVcv.ps    

\begin{document}

\def\heading{\noindent}
\def\ref{\vskip -0.05in \hangindent 2.5pc \hangafter 1}
\def\listing{\vskip -0.1in \hangindent 2.5pc \hangafter 1}
\def\Bline#1{\hskip 0.3in $\bullet$ #1 \\ \smallskip}
\def\Sline#1{\hskip 0.6in #1 \\}
\def\line#1{#1 \\}


\centerline{\bf Biographical Sketch for Daniel Buscombe}

\heading {\bf  Professional Preparation}

\hskip -0.1in \begin{tabular}{llll}
  University of Plymouth &  Nearshore Oceanography        & Ph.D.   & 2004 - 2008 \\
  Lancaster University           &  Physical Geography & B.Sc.   & 2000 - 2003 \\
\end{tabular}

\heading {\bf  Appointments and Teaching Experience}

\hskip -0.1in \begin{tabular}{lll}
  U.S Geological Survey    &  Research Geologist & 2012 - Present \\
  University of Plymouth &  NERC Postdoctoral Research Fellow & 2009 - 2012\\
  UC Santa Cruz &  Postdoctoral Research Fellow & 2008 - 2009 \\
  University of Plymouth &  Research Assistant & 2007 - 2008\\
  University of Plymouth &  Teaching Assistant & 2004 - 2008\\
  Field Studies Council & Assistant Tutor & 2003 - 2004\\
\end{tabular}

\heading {\bf Selected Related Publications} 

%Buscombe, D., Grams, P.E., Kaplinski, M.A., 2014, Characterizing riverbed sediment using high-frequency acoustics 2: Scattering signatures of Colorado River bed sediment in Marble and Grand Canyons. Journal of Geophysical Research - Earth Surface 119, doi:10.1002/2014JF003191
%Buscombe, D., 2013, Transferable Wavelet Method for Grain Size-Distribution from Images of Sediment Surfaces and Thin Sections, and Other Natural Granular Patterns. Sedimentology 60: 1709–1732
%Lacy, J.R., Rubin, D.M. and Buscombe, D., 2012, Currents, Drag and Sediment Transport Induced by a Tsunami. Journal of Geophysical Research - Oceans, 117, C09028
%Austin, M.J., and Buscombe, D., 2008, Morphological Change and Sediment Dynamics of the Beach Step on a Macrotidal Gravel Beach. Marine Geology 249, 167-183
%Buscombe, D. and Masselink, G., 2006. Concepts in gravel beach dynamics. Earth Science Reviews 79, 33-52.

\ref
 {\bf Jake Vanderplas}, Andrew Connolly, \v{Z}eljko Ivezi\'{c}, \& Alex Gray. {\it AstroML: Machine Learning for Astronomy in Astrophysics}. CIDU proceedings, 2012

\ref
 {\bf Jake Vanderplas}, Andrew Connolly, Bhuvnesh Jain, \& Mike Jarvis. {\it Interpolating Masked Weak Lensing Signals with Karhunen-Loeve Analysis}. ApJ 744:180, 2012.

\ref
 {\bf Jake Vanderplas}, Andrew Connolly, Bhuvnesh Jain, \& Mike Jarvis. {\it 3D Reconstruction of the Density Field: An SVD Approach to Weak Lensing Tomography}. ApJ 727:118, 2011.

\ref
 {\it LSST Science Collaboration LSST Science Book, Version 2.0}, 2009 arXiv:0912.0201

\ref
 {\bf Jake Vanderplas} \& Andrew Connolly. {\it Reducing the Dimensionality of Data: Locally Linear Embedding of Sloan Galaxy Spectra}. AJ 138:1365, 2009.

\heading {\bf Other Significant Publications}

%Buscombe, D., Rubin, D.M., Lacy, J.R., Storlazzi, C., Hatcher, G., Chezar, H., Wyland, R., and Sherwood, C., 2014, Autonomous bed-sediment imaging-systems for revealing temporal variability of grain size. Limnology and Oceanography: Methods, 12, 390 – 406
%Williams, J.J., Buscombe, D., Masselink, G., Turner, I., and Swinkels, C., 2012, Barrier Dynamics Experiment (BARDEX): Aims, Design and Procedures. Coastal Engineering 63, 3-12.
%Buscombe, D., and Conley, D.C., 2012, Effective Shear Stress of Graded Sediment. Water Resources Research, 48, W05506
%Buscombe, D., Rubin, D.M., and Warrick, J.A., 2010, Universal Approximation of Grain Size from Images of Non-Cohesive Sediment. Journal of Geophysical Research - Earth Surface 115, F02015
%Masselink, G., Buscombe, D., Austin, M.J, O’Hare, T., Russell, P., 2008, Sediment Trend Models Fail to Reproduce Small Scale Sediment Transport Patterns on an Intertidal Beach. Sedimentology 55, 667-687.

\ref
V. Vikram, A. Cabre, B. Jain \& {\bf J. VanderPlas}. {\it Astrophysical Tests of Modified Gravity: the Morphology and Kinematics of Dwarf Galaxies}. JCAP 08:20, 2013

\ref
Bhuvnesh Jain \& {\bf Jake Vanderplas}. {\it Tests of Modified Gravity with Dwarf Galaxies}. JCAP 10:32, 2011.

\ref
Scott Daniel, Andrew Connolly, Jeff Schneider, {\bf Jake Vanderplas} \& Liang Xiong. Classification of Stellar Spectra with LLE. AJ 142:203, 2011.

\ref
Pedregosa, F.; Varoquaux, G.; Gramfort, A.; Michel, V.; Thirion, B.; Grisel, O.; Blondel, M.; Prettenhofer, P.; Weiss, R.; Dubourg, V.; {\bf Vanderplas, J.}; Passos, A.; Cournapeau, D.; Brucher, M.; Perrot, M.; Duchesnay, E. {\it Scikit-learn: Machine learning in Python}. Journal of Machine Learning Research, 12:2825, 2011

\ref
R. Kessler, A. Becker, D. Cinabro, {\bf J. Vanderplas}, \& 42 co-authors. {\it First-Year Sloan Digital Sky Survey-II Supernova Results: Hubble Diagram and Cosmological Parameters}. ApJ 703:1374, 2009.

\heading {\bf Synergistic Activities}

%I am committed in both my research and teaching activities to the study of process sedimentology, acoustic remote sensing, sediment transport and morphodynamics. (1) I’m involved in casual teaching at NAU, and I’m currently training and advising one PhD and two Masters students in my research program. (2) I’m part of the steering committee of an informal group of multibeam sonar users (from industry, academia, and other agencies) in shallow water environments (https://sites.google.com/site/mbesinriverssummitworkshop/). (3) I’m leading several open-source community software development initiatives for processing data from sidescan sonar, multibeam sonar, photogrammetry, holographic imagery, sediment imagery, point clouds, and many more (https://github.com/dbuscombe-usgs), motivated by a passion for transparency and reproducibility in data analysis and research, using open-source tools. (4) Convened Session at Fall 2007 AGU Conference on ‘Linking sediment supply, bed-sediment particle size, sediment transport and bed morphology in fluvial, marine, and Aeolian settings’. (5) Convened Session at Fall 2013 AGU Conference on ‘Fluvial sediment budgets: can we do better?’ (6) Convened the Young Coastal Scientists and Engineers Conference, Plymouth University 2007. (7) Ad hoc reviewer for 12 journals, and 2 science research councils.

\listing
{\bf Open Source Contributions:} I have developed fast sparse matrix eigen-decomposition code and graphical analysis for the numerical package {\tt scipy}, and a number of optimized supervised and unsupervised learning and data visualization methods for the machine learning packages {\tt scikit-learn} and {\tt MDP-toolkit}. I created {\tt SciDB-py}, a Python interface to SciDB.  I have also written and contributed to astronomy-specific packages such as {\tt astroML} (Astronomy Machine Learning), and {\tt SNANA} (Fermilab's supernova analysis software).

\listing
{\bf Digital Planetarium:} From 2010-2011, I managed and coordinated the upgrade of the UW planetarium to a digital system based on the World Wide Telescope software. This was a joint project between the University of Washington and Microsoft Research. I have developed related educational tools for K-12 class visits as well as undergraduate astronomy courses. I have occasionally partnered with Microsoft as an astronomy expert at a variety of education and technology conferences around the country.

\listing
{\bf Science Communication Fellow:} I have participated in the Portal to the Public training program at the Pacific Science Center, and have volunteered regularly since 2009 as a Science Communication Fellow, exploring astronomical research with visitors to the museum.

\listing
{\bf Undergraduate Mentoring:} I have participated as a mentor for the U. Washington PreMajor in Astronomy Program (PreMAP), providing research experiences for under-graduates from demographics which are traditionally under-represented in the sciences.

\listing
{\bf K-12 Curriculum Development:} I taught for two years at the Mount Hermon Outdoor Science School, where among other activities I developed an outdoor astronomy curriculum for K-12 students, and conducted an astronomy training workshop for my peers at a regional conference for outdoor educators.

\heading {\bf Selected Collaborators}

 \hskip -0.1in \begin{tabular}{ll}
Paul Grams, {\it U.S. Geological Survey} & Jon Warrick, {\it U.S. Geological Survey}\\
David Rubin, {\it University of California Santa Cruz} & David Topping, {\it U.S. Geological Survey}\\
Daniel Conley, {\it University of Plymouth} & Joe Wheaton, {\it Utah State University}\\
Alex Nimmo-Smith, {\it University of Plymouth} & Jack Puleo, {\it University of Delaware}\\
Jessie Lacy, {\it U.S. Geological Survey} & Sean Smith, {\it University of Maine}\\
Emlyn Davies, {\it SINTEF, Norway} & Gerd Masselink, {\it University of Plymouth}\\
Martin Austin, {\it Bangor University} & Brandon McElroy, {\it University of Wyoming}\\
\end{tabular}

\heading {\bf Graduate Advisors}

\hskip -0.1in \begin{tabular}{lll}
  Gerd Masselink & {\it University of Plymouth} & (2004-2008)\\
  Mark Davidson & {\it University of Plymouth} & (2004-2008)\\
\end{tabular}

\end{document}
