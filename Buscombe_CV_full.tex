
%________________________________________________________________________________________
% @brief    LaTeX2e Resume for Daniel Buscombe
% @author   Daniel Buscombe
% @date     February 2014

%________________________________________________________________________________________
\documentclass[margin,line]{resume}

\usepackage{url}

\begin{document}
\name{\Large Daniel Buscombe}
\begin{resume}

Research Geologist, United States Geological Survey,\\
Grand Canyon Monitoring and Research Center, 2255 N Gemini Drive, Flagstaff, AZ 86001 \\
http://dbuscombe-usgs.github.com

    %____________________________________________________________________________________
%    % Contact Information
    \section{\mysidestyle Personal\\Details}\vspace{2mm}

    \begin{tabular}{@{} l @{\hspace{55mm}} r}
    Place of Birth: & Greenwich, London, UK (3 December 1981) \\
    Nationality: & British 
    \end{tabular}

%    %____________________________________________________________________________________
    % Contact Information
    \section{\mysidestyle Contact\\Information}\vspace{2mm}

    \begin{tabular}{@{} l @{\hspace{20mm}} r}
    Tel: 928-726-7216 & email: dbuscombe@usgs.gov \\
    \end{tabular}

    %____________________________________________________________________________________
    % Research Interests
\section{\mysidestyle Research\\Interests}
     I am a geomorphologist/sedimentologist with a principal research interest in sediment dynamics, which includes the physical and biochemical makeup of sediment beds in a range of aquatic environments, the role of sediment heterogeneity in the physics of sediment transport, and the morphodynamics (both micro- and macro-scale) that result from sediment transport, sedimentation, stratigraphy and geomorphological forms. I study the complex inter-relations between fluid flows, geomorphology, sediment transport and sedimentology. I investigate these processes by developing novel field-deployed optical and acoustic imaging systems, and computational algorithms for small-scale sediment hydroacoustics, in-situ particle and bed imaging and flow-field/turbulence measurements. 


    %____________________________________________________________________________________
    % Education
    \section{\mysidestyle Education}
    \begin{footnotesize}

    {\bf Ph.D. (2008), Coastal Geomorphology/Nearshore Oceanography, University of Plymouth}, Plymouth, UK. {\sl Morphodynamics, Sediment Dynamics and Sedimentation of a Gravel Beach}. Advisor:  Prof. Gerhard Masselink.\vspace{2mm}

%    \vspace*{2mm}
    {\bf BSc (Hons), 1st class (2003), Physical Geography with Minors in Environmental Sciences and Biology, Lancaster University}, Lancaster, UK.  {\sl Morphodynamics of a Ridge-and-Runnel System on a Macrotidal Beach}. Advisor:  Dr Suzanna Ilic. \vspace{2mm}
     \end{footnotesize}

    %____________________________________________________________________________________
    % Professional Experience
    \section{\mysidestyle Employment\\History}
    \begin{footnotesize}

    {\bf November 2012 --  present}. {\sl Research Geologist, Grand Canyon Monitoring and Research Center, U.S. Geological Survey, Flagstaff, AZ, USA.}

    %{\bf University of Plymouth}, Plymouth, United Kingdom \vspace{2mm}\\\vspace{1mm}%
    {\bf October 2009 -- November 2012}. {\sl Post-doctoral Research Fellow, School of Marine Science \& Engineering, University of Plymouth, UK.} %TSSAR Waves (Turbulence, Sediment Stratification and Altered Resuspension under Waves), a UK research council funded project. Advisors: Dr Daniel C. Conley and Dr Alex Nimmo-Smith.

    %{\bf University of Plymouth}, Plymouth, United Kingdom \vspace{2mm}\\\vspace{1mm}%
    {\bf September, 2008 -- 2011}. {\sl Computer Programming Contractor, Marine Biology \& Ecology Research Centre, University of Plymouth, UK.} %Underwater image perspective transformations and GUI development work for a team of deep-sea ecologists led by Dr Kerry Howell.

    %{\bf University of California}, Santa Cruz, United States \vspace{2mm}\\\vspace{1mm}%
    {\bf October, 2008 -- October 2009}. {\sl Post-doctoral Research Scholar, United States Geological Survey, Santa Cruz, California, USA.} %Santa Cruz Seafloor Observatory Project. Advisors: Dr David M. Rubin and Dr Jessica Lacy.

    %{\bf University of Plymouth}, Plymouth, United Kingdom \vspace{2mm}\\\vspace{1mm}%
    {\bf June, 2008 -- September, 2008}. {\sl Research Assistant, School of Geography, University of Plymouth, UK.} %BARDEX (Barrier Dynamics Experiment), an EU Hydralab III-funded laboratory wave flume project. Advisors: Prof. Jon Williams and Prof. Gerhard Masselink.

    %{\bf University of Plymouth}, Plymouth, United Kingdom \vspace{2mm}\\\vspace{1mm}%
    {\bf December, 2007 -- April, 2008}. {\sl Research Assistant, School of Earth, Ocean \& Environmental Science, University of Plymouth, UK.} %WHISSP (Wave Hub Impacts on Seabed and Shoreline Processes), an EU-funded field-based project studying the effects of marine renewable energy devices on the shoreline. Advisors: Prof. Jon Williams and Prof. Gerhard Masselink.

    %{\bf University of Plymouth}, Plymouth, United Kingdom \vspace{2mm}\\\vspace{1mm}%
    {\bf October, 2004 -- July 2008}. {\sl Associate Lecturer and Demonstrator (part-time), School of Geography, University of Plymouth, UK.}
%    Modules on coastal, fluvial and glacial geomorphology

    {\bf August 2003 - September, 2004}. {\sl Assistant tutor, Field Studies Council, Castle Head, Grange-over-Sands, UK}. %Teaching Ecology and Physical Geography at a residential field centre to groups of all ages
        \end{footnotesize}

    %____________________________________________________________________________________
    % Publications
    \section{\mysidestyle Journal Publications}

	\subsection{\mysidestyle 2006}
        \begin{footnotesize}
	\begin{list1}
	\item[1] {\bf Buscombe, D.}, and Masselink, G. (2006) Concepts in Gravel Beach Dynamics. {\sl Earth Science Reviews} 79, 33-52.
	\end{list1}
	
	\subsection{\mysidestyle 2008}
	\begin{list1}
	\item[2] Masselink, G., {\bf Buscombe, D.}, Austin, M.J, O'Hare, T., Russell, P. (2008) Sediment Trend Models Fail to Reproduce Small Scale Sediment Transport Patterns on an Intertidal Beach. {\sl Sedimentology} 55, 667-687.\\
	
	\item[3] Austin, M.J., and {\bf Buscombe, D.} (2008) Morphological Change and Sediment Dynamics of the Beach Step on a Macrotidal Gravel Beach. {\sl Marine Geology} 249, 167-183. \\
	
	\item[4] {\bf Buscombe, D.} (2008) Estimation of Grain Size Distributions and Associated Parameters from Digital Images of Sediment. {\sl Sedimentary Geology}  210, 1-10.\\

	\item[5] Masselink, G., and {\bf Buscombe, D.} (2008) Shifting gravel: A case study of Slapton Sands. {\sl Geography Review} 22 (1), 27-31.
	\end{list1}

	\subsection{\mysidestyle 2009}
	\begin{list1}
	\item[6] {\bf Buscombe, D.}, and Masselink, G. (2009) Grain Size Information from the Statistical Properties of Digital Images of Sediment. {\sl Sedimentology} 56, 421-438 \\

	\item[7] Warrick, J.A., Rubin, D.M., Ruggiero, P., Harney, J., Draut, A.E., and {\bf Buscombe, D.} (2009) Cobble Cam: Grain-size measurements of sand to boulder from digital photographs and autocorrelation analyses. {\sl Earth Surface Processes and Landforms} 34, 1811-1821.\\

	\item[8] Williams, J., Masselink, G., {\bf Buscombe, D.}, Turner, I., Matias, A., Ferreira, O., Meltje, N., Bradbury, A., Albers, T., and Pan, S. (2009) BARDEX (Barrier Dynamics Experiment): taking the beach into the laboratory. {\sl Journal of Coastal Research} SI 56, 158-162.
	\end{list1}

	\subsection{\mysidestyle 2010}
	\begin{list1}
	\item[9] {\bf Buscombe, D.}, Rubin, D.M., and Warrick, J.A. (2010) Universal Approximation of Grain Size from Images of Non-Cohesive Sediment. {\sl Journal of Geophysical Research - Earth Surface} 115, F02015.
	\end{list1}

	\subsection{\mysidestyle 2012}
	\begin{list1}
	\item[10] Williams, J.J., {\bf Buscombe, D.}, Masselink, G., Turner, I., and Swinkels, C. (2012) Barrier Dynamics Experiment (BARDEX): Aims, Design and Procedures. {\sl Coastal Engineering} 63, 3-12.\\

	\item[11] {\bf Buscombe, D.}, and Conley, D.C. (2012) Effective Shear Stress of Graded Sediment. {\sl Water Resources Research} 48, W05506.\\

	\item[12] {\bf Buscombe, D.}, and Rubin, D.M. (2012) Advances in the Simulation and Automated Measurement of Granular Material, Part 1: Simulations. {\sl Journal of Geophysical Research - Earth Surface} 117, F02001.\\

	\item[13] {\bf Buscombe, D.}, and Rubin, D.M. (2012) Advances in the Simulation and Automated Measurement of Granular Material, Part 2: Direct Measures of Particle Properties. {\sl Journal of Geophysical Research - Earth Surface} 117, F02002.\\

	\item[14] Lacy, J.R., Rubin, D.M. and {\bf Buscombe, D.} (2012) Currents and sediment transport induced by a tsunami far from its source. {\sl Journal of Geophysical Research - Oceans} 117, C09028.
	\end{list1}

	\subsection{\mysidestyle 2013}
	\begin{list1}
        \item[15] {\bf Buscombe, D.} (2013) Transferable Wavelet Method for Grain Size-Distribution from Images of Sediment Surfaces and Thin Sections, and Other Natural Granular Patterns. {\sl Sedimentology} 60, 1709--1732. 

	\end{list1}

	\subsection{\mysidestyle 2014}
	\begin{list1}
        \item[16] Puleo, J., Blenkinsopp, C., Conley, D., Masselink, G., Turner, I., Russell, P., {\bf Buscombe, D.}, Howe, D., Lanckriet, T., McCall, R., and Poate, T. (2014) A Comprehensive Field Study of Swash-Zone Processes, Part 1: Experimental Design with Examples of Hydrodynamic and Sediment Transport Measurements. {\sl Journal of Waterway, Port, Coastal, and Ocean Engineering}, 140, 29–42. 10.1061/(ASCE)WW.1943-5460.0000210.\\

	\item[17]  {\bf Buscombe, D.}, Rubin, D.M., Lacy, J.R., Storlazzi, C., Hatcher, G., Chezar, H., Wyland, R. and Sherwood, C. (2014) Autonomous bed-sediment imaging-systems for revealing temporal variability of grain size. {\sl Limnology and Oceanography: Methods, 12, 390 - 406}

	\item[18] {\bf Buscombe, D.}, Grams, P.E., Kaplinski, M.A. (2014), Characterizing riverbed sediment using high-frequency acoustics 1: Spectral properties of scattering. {\sl Journal of Geophysical Research - Earth Surface, 119, doi:10.1002/2014JF003189.}.\\

	\item[19] {\bf Buscombe, D.}, Grams, P.E., Kaplinski, M.A. (2014), Characterizing riverbed sediment using high-frequency acoustics 2: Scattering signatures of Colorado River bed sediment in Marble and Grand Canyons. {\sl Journal of Geophysical Research - Earth Surface, 119, doi:10.1002/2014JF003191}.\\

	\end{list1}
	
	\subsection{\mysidestyle 2015}
	\begin{list1}
	
	\item[20] Davies, E.J., {\bf Buscombe, D.}, Graham, G.W., Nimmo-Smith, W.A.M., 2015, Evaluating unsupervised methods to size and classify suspended particles using digital in-line holography, {\sl Journal of Atmospheric \& Oceanographic Technology}, accepted.

	\end{list1}

	\subsection{\mysidestyle In review/preparation}
	\begin{list1}

	\item[21] {\bf Buscombe, D.}, Grams, P.E., Smith, S.M., in review, Automated riverbed sediment classification using low-cost sidescan sonar. {\sl Journal of Hydraulic Engineering}.\\

	\item[22] {\bf 	Buscombe, D.}, Conley, D.C., and Nimmo-Smith, W.A.M., in prep., Sorting in the Surf Zone on an Energetic Sand Beach Revealed by Acoustic and Holographic Measurements of Suspended Sediment. Intended for {\sl Marine Geology}\\

	\item[23] {\bf 	Buscombe, D.}, in prep., Computational considerations for spectral analysis of point clouds for texture segmentation. Intended for {\sl Computers \& Geosciences}

%	\item[19] {\bf Buscombe, D.}, and Nimmo-Smith, W.A.M. (in prep.) Field video observations of gravel motion under near-breaking waves. Intended for {\sl Journal of Geophysical Research - Oceans}.

%	\item[16] {\bf Buscombe, D.}, Conley, D.C., and Rubin, D.M. (in prep.) Sorting on a sand beach alternating between intermediate and reflective states. Intended for {\sl Marine Geology}.

	\end{list1}
        
        \end{footnotesize}

    \section{\mysidestyle Conference Publications}

	\subsection{\mysidestyle 2007}
        \begin{footnotesize}
	\begin{list1}
	\item[1] {\bf Buscombe, D.}, Austin, M.J., and Masselink, G. (2007) Field observations of step dynamics on a macrotidal gravel beach. In Kraus, N., and Rosati, J., (Eds) {\sl Proceedings of Coastal Sediments 2007 (Volume 1)}, ASCE, USA (oral).\\
	
	\item[2] {\bf Buscombe, D.}, and Masselink, G. (2007) The relationship between sediment properties and sedimentation patterns on a macrotidal gravel beach over a semi lunar tidal cycle. {\sl Eos Transactions American Geophysical Union Fall Meeting}, Abstract H53L-02 (oral).
	\end{list1}

	\subsection{\mysidestyle 2008}
	\begin{list1}
	\item[3] {\bf Buscombe, D.}, Masselink, G., and Rubin, D.M. (2008) Granular Properties from Digital Images of Sediment: Implications for Coastal Sediment Transport Modelling. {\sl International Conference on Coastal Engineering (ICCE)}, Hamburg, 2008 (oral).\\
	
	\item[4] Ruiz de Alegria, A., Masselink, G., Kingston, K., Williams, J., and {\bf Buscombe, D.} (2008) Storm Impacts on a Gravel Beach Using the ARGUS video system. {\sl International Conference on Coastal Engineering (ICCE)}, Hamburg, 2008 (oral).\\
	
	\item[5] Austin, M.J., Masselink, G., Turner, I., {\bf Buscombe, D.}, and Williams, J. (2008) Groundwater seepage between a gravel barrier beach and a freshwater lagoon. {\sl International Conference on Coastal Engineering (ICCE)}, Hamburg, 2008 (oral).\\
	
	\item[6] {\bf Buscombe, D.}, Ruiz de Alegria, A., and Masselink, G. (2008). The relative importance of cross- and along-shore sediment transport in planform and profile adjustments of a gravel barrier beach: Slapton, Devon, UK. {\sl American Geophysical Union Fall Meeting}, San Francisco, Dec 2008 (poster).
	\end{list1}

	\subsection{\mysidestyle 2009}
	\begin{list1}
	\item[7] Williams, J.J., Masselink, G., {\bf Buscombe, D.}, and 7 others (2009). BARDEX (Barrier Dynamics Experiment): taking the beach into the laboratory. Abstract submitted for oral presentation at the {\sl 10th International Coastal Symposium (ICS)}, Lisbon, Portugal 2009 (oral).
	\end{list1}

	\subsection{\mysidestyle 2010}
	\begin{list1}
	\item[8] {\bf Buscombe, D.} Lacy, J.R., and Rubin, D.M. (2010) Fractional resuspension and sediment flux on a wave-dominated, non-cohesive, inner continental shelf. {\sl Ocean Sciences 2010}, Portland (poster)\\

	\item[9] Rubin, D.M., {\bf Buscombe, D.}, Lacy, J.R., Chezar, H., Hatcher, G., and Wyland, R. (2010) Seafloor sediment observatory on a cable and a shoestring. {\sl Ocean Sciences 2010}, Portland (oral)\\

	\item[10] {\bf Buscombe, D.}, and Conley, D.C. (2010) Modeling sand resuspension and stratification in turbulent nearshore flows: sensitivity to grain size distribution. {\sl Ocean Sciences 2010}, Portland (oral)\\

	\item[11] Lacy, J.R., {\bf Buscombe, D.}, and Rubin, D.M. (2010) Tsunami-enhanced sediment resuspension on the inner shelf in northern Monterey Bay, California. {\sl Ocean Sciences 2010}, Portland (oral)\\

	\item[12] Conley, D.C., and {\bf Buscombe, D.} (2010) Effects of Grain Size Distributions on Fluid-Sediment Feedback. {\sl European Geosciences Union General Assembly 2010}, Vienna (oral)\\

	\item[13] {\bf Buscombe, D.}, Rubin, D. M., and Warrick, J. A. (2010) An automated and 'universal' method for measuring mean grain size from a digital image of sediment. {\sl 9th Federal Interagency Sedimentation Conference}, Las Vegas June 2010 (oral).\\

	\item[14] Rubin, D.M., Chezar, H., {\bf Buscombe, D.}, Warrick, J.A., Barnard, P.L., Lacy, J.R., Hatcher, G., Wyland, R., Storlazzi, C., Conaway, C.H., Topping, D.J., Melis, T.S., and Grams, P.E. (2010) New technology for in-situ grain-size analysis from digital images of sediment, and resulting insights regarding sediment transport.  {\sl 9th Federal Interagency Sedimentation Conference}, Las Vegas June 2010 (oral).\\

	\item[15] {\bf Buscombe, D.}, Rubin, D.M., and Lacy, J.R. (2010) Hourly Measurements of Grain-Size from the Inner Continental Shelf Seabed Using a Fully-Automated, Hydraulically-Controlled Underwater Video Microscope. {\sl Particles in Europe 2010}, Villefranche-Sur-Mer, France. (oral)\\

	\item[16] Williams, J.J., Masselink, G., {\bf Buscombe, D.}, and 10 others (2010) BARDEX (Barrier Dynamics Experiments): a laboratory study of gravel barrier response to waves and tides. {\sl Proceedings of Hydralab III Joint User Meeting}, Hannover, p. 4 (oral)
	\end{list1}

	\subsection{\mysidestyle 2011}
	\begin{list1}
	\item[17] {\bf Buscombe, D.}, and Conley, D.C. (2011) Formula for Motion Threshold per Grain Size for Graded Sediments in Steady Flows. {\sl European Geosciences Union General Assembly 2011}, Vienna (poster).\\

        \item[18] {\bf Buscombe, D.}, and Rubin, D.M. (2011) How do you tell how big something is without direct measurement? Estimating grain size using an image’s spectrum. {\sl American Geophysical Union Fall Meeting}, San Francisco, Dec 2011 (oral).
	\end{list1}

	\subsection{\mysidestyle 2012}
	\begin{list1}
        \item[19] Conley, D.C., {\bf Buscombe, D.}, and Nimmo-Smith, A. (2012) New understandings of sediment suspension in the nearshore from cross-comparisons of diverse  sensors. {\sl Ocean Sciences 2012}, Salt Lake City (poster).\\

	\item[20] {\bf Buscombe, D.}, Conley, D.C., and Rubin, D.M. (2012) Co-variation of intertidal morphology, bedforms and grain size on a macrotidal sand beach: Praa Sands, UK. {\sl Ocean Sciences 2012}, Salt Lake City (oral).\\

	\item[21] Puleo, J.A., Conley, D.C., Masselink, G., Russell, P., Turner, I.L., Blenkinsopp, C., {\bf Buscombe, D.}, Lanckriet, T., McCall, R., and Poate, T. (2012) Comprehensive study of swash-zone hydrodynamics and sediment transport. {\sl International Conference on Coastal Engineering}, Santander, July 2012 (oral).\\

	\item[22] {\bf Buscombe, D.}, and Conley, D.C. (2012) Schmidt number of sand suspensions under oscillating-grid turbulence. {\sl International Conference on Coastal Engineering}, Santander, July 2012 (oral).\\

	\item[23] Conley, D.C., {\bf Buscombe, D.}, and Nimmo-Smith, A. (2012) Use of digital holographic cameras to examine the measurement and understanding of sediment suspension in the nearshore. {\sl International Conference on Coastal Engineering}, Santander, July 2012 (oral).\\

	\item[24] Nimmo-Smith, A., {\bf Buscombe, D.}, and Conley, D.C. (2012) Use of digital holographic cameras to examine the measurement and understanding of sediment suspension in the nearshore. {\sl Particles in Europe}, Barcelona, October 2012 (oral).
	\end{list1}

	\subsection{\mysidestyle 2013}
	\begin{list1}
	\item[25] Kaplinski, M.A., Hazel, J.E., Grams. P.E., {\bf Buscombe, D.}, Hadley, D., and Kohl. K. (2013) Constructing a morphologic sediment budget, with uncertainties, for a 50-km segment of the Colorado River in Grand Canyon.  {\sl American Geophysical Union Fall Meeting}, San Francisco, Dec 2013 (poster).\\

	\item[26] Grams. P.E., {\bf Buscombe, D.}, Hazel, J.E., Kaplinski, M.A., and Topping, D.J. (2013) Reconciliation of Flux-based and Morphologic-based Sediment Budgets. {\sl American Geophysical Union Fall Meeting}, San Francisco, Dec 2013 (oral). \\

	\item[27] {\bf Buscombe, D.}, Grams. P.E., Kaplinski, M.A. (2013) Acoustic Scattering by an Heterogeneous River Bed: Relationship to Bathymetry and Implications for Sediment Classification using Multibeam Echosounder Data. {\sl American Geophysical Union Fall Meeting}, San Francisco, Dec 2013 (oral). \\

	\item[28] Davies, E.J., {\bf Buscombe, D.}, Graham, G., Nimmo Smith, W.A.M. (2013) Evaluating Unsupervised Methods to Size and Classify Suspended Particles Using Digital Holography {\sl American Geophysical Union Fall Meeting}, San Francisco, Dec 2013 (poster). 
	\end{list1}

	\subsection{\mysidestyle 2014}
	\begin{list1}
        \item[29] Rubin, D., Topping, D., Grams, P., Tusso, R., Schmidt, J., {\bf Buscombe, D.}, Melis, T., Wright, S. (2014) What sediment grain size reveals about suspended-sediment transport in the Colorado River in Grand Canyon. {\sl International Conference on the Status and Future of the World's Large Rivers}, Brazil (oral).\\

        \item[30] {\bf Buscombe, D.}, Grams. P.E., and Kaplinski, M.A. (2014) Bed sediment classification using acoustic backscatter {\sl 1st MBES in Rivers Workshop}, Utah State University, Feb 2014. (oral)\\

        \item[31] {\bf Buscombe, D.}, Grams. P.E. (2014) Topographic and acoustic estimates of grain-scale roughness from high-resolution multibeam echo-sounder: examples from the Colorado River in Marble and Grand Canyons. {\sl American Geophysical Union Fall Meeting}, San Francisco, Dec 2014. (oral)
	\end{list1}

	\subsection{\mysidestyle 2015}
	\begin{list1}
	
        \item[32] {\bf Buscombe, D.}, Grams. P.E., and Kaplinski, M.A. (2015) Towards Sedimentary Change Detection in Lower Marble Canyon using Acoustic Sediment Classification {\sl 2nd MBES in Rivers Workshop}, USGS Flagstaff, AZ, March 2015. (oral)\\
        
        \item[33] Grams. P.E., {\bf Buscombe, D.}, Topping, D.J., Hazel, J.E., and Kaplinski, M.A. (2015) Use of Flux and Morphologic Sediment Budgets for Sandbar Monitoring on the Colorado River in Marble Canyon, Arizona. {\sl 10th Federal Interagency Sedimentation Conference}, Reno, April 2015 (oral).\\

        \item[34] {\bf Buscombe, D.}, Grams. P.E., Kaplinski, M.A., Tusso, R.B., and Rubin, D.M. (2015) Hydroacoustic signatures of Colorado riverbed sediments in Marble and Grand Canyons using multibeam sonar. {\sl 10th Federal Interagency Sedimentation Conference}, Reno, April 2015 (oral).\\

        \item[35] {\bf Buscombe, D.}, Grams. P.E., Melis, T.S., Smith, S. (2015) Considerations for unsupervised riverbed sediment characterization using low-cost sidescan sonar: Examples from the Colorado River, AZ and the Penobscot River, ME. {\sl 10th Federal Interagency Sedimentation Conference}, Reno, April 2015 (oral).\\

        \item[36] {\bf Buscombe, D.}, Tusso, R.B., Grams. P.E. (2015) Using oblique digital photography for alluvial sandbar monitoring and low-cost change detection. {\sl 10th Federal Interagency Sedimentation Conference}, Reno, April 2015 (oral).

	\end{list1}

        \end{footnotesize}

    %____________________________________________________________________________________
    % Reports
	\subsection{\mysidestyle Reports}
        \begin{footnotesize}
	\begin{list1}
	 
	\item[1] {\bf Buscombe, D.}, and Scott, T.M. (2008) {\sl Coastal Geomorphology of North Cornwall: St Ives to Trevose Head}. Internal report for Wave Hub Impacts on Seabed and Shoreline Processes, University of Plymouth. 170pp.\\
	\item[2] {\bf Buscombe, D.}, Williams, J. J., and Masselink, G. (2008) {\sl BARDEX (Barrier Dynamics Experiment): experimental procedure, technical information and data report}. Technical report for the European Union Hydralab III, 219pp. 

	\end{list1}
        \end{footnotesize}

	\subsection{\mysidestyle Software}

        \begin{footnotesize}
	\begin{list1}
	 
	\item[1] {\bf Digital Grain Size.} Software for automated analyses of grain size from images of sediment. Source code currently available in Matlab and Python. Webpage \url{http://dbuscombe-usgs.github.com}\\
	\item[2] {\bf PyHum.} Software for reading, processing and analysis of Humminbird sidescan data. Source code available in Python/Cython. Webpage \url{http://dbuscombe-usgs.github.com}\\
	\item[3] {\bf Benthic Analysis Tool.} Software for the semi-automation of species identification and measurement in deep-sea ROV/drop frame images. Source code available in Matlab.\\
        \item[4] {\bf Sand Simulation Toolbox.} Software for generating 3D discrete particle models consisting of realistic particles (with a size- and shape-distribution) with user-defined properties. Source code available in Matlab. Webpage \url{http://dbuscombe-usgs.github.com}\\
        \item[5] {\bf MATSCAT.} Software for analysis of multiple-frequency acoustic backscatter for suspended sediment concentration and particle size. Source code available in Matlab.\\
      \item[6] Generic software for serial data acquisition and real-time display. Source code available in Python.\\
        \item[7] Software for interfacing with machine-vision ethernet video cameras. Source code available in C.

	\end{list1}
        \end{footnotesize}

    %____________________________________________________________________________________
%    % Grants
    \section{\mysidestyle Funded \\ Proposals}
        \begin{footnotesize}
	\begin{list1}
	\item[1] {\sl British Geomorphological Society Postgraduate award} ($\pounds$300) to attend and present at Coastal Sediments 2007, in New Orleans, USA\\
	
	\item[2] {\sl American Geophysical Union Student travel grant} ($\$$600) to attend the AGU 2007 Fall Meeting in San Francisco, USA\\
	
	\item[3] {\sl International Association of Sedimentologists Grant} (700 euros) to investigate nearshore bedload transport and bedforms with stereo underwater video cameras.\\
	
	\item[4] {\sl International Association for Mathematical Geology research grant} ($\$$2000) to develop and trial algorithms for quantification of granular properties and coarse-grain sediment transport from images of the sea bed\\
	
	\item[5] {\sl Society for Sedimentary Geology Grant} ($\$$500, President's Fund) to investigate nearshore bedload transport and bedforms with stereo underwater video cameras.\\
	
	\item[6] {\sl Challenger Society for Marine Science travel grant} ($\pounds$150) to attend and present at ICCE Hamburg 2008\\
	
	\item[7] {\sl Plymouth Marine Science Education Fund} ($\pounds$250) to attend and present at ICCE Hamburg 2008\\

	\item[8] Co-Investigator; G. Masselink (PI), D.C. Conley, D. Buscombe., (2012 - 2014) {\sl Proto-type Experiment and Numerical Modelling of Energetic Sediment Transport under Waves (PESTS)}. Engineering and Physical Sciences Research Council, UK. EPSRC EP/K000306/1 ($\pounds$240,000)\\

	\item[9] Co-Investigator (multiple PIs – J. Schmidt and others), (2013 - 2014) {\sl Sandbars and sediment storage dynamics: long-term monitoring and research at the site, research and ecosystem scales}, Grand Canyon Monitoring and Research Center Biennial Work Plan. Glen Canyon Dam Adaptive Management Work Group ($\$$2,911,400)\\

	\item[10] Co-Investigator (multiple PIs – P.E. Grams and others), (2014 - 2017) {\sl Geomorphic Processes and Relations Among Flow Regime, Sediment Flux and Resource Conditions on the Green River in Canyonlands National Park}. National Park Service ($\$$232,016)\\

	\item[11] Co-Investigator (multiple PIs – J. Schmidt and others), (2015 - 2017) {\sl Sandbars and sediment storage dynamics: long-term monitoring and research at the site, research and ecosystem scales}, Grand Canyon Monitoring and Research Center Triennial Work Plan. Glen Canyon Dam Adaptive Management Work Group ($\$$4,253,400).

	\end{list1}
        \end{footnotesize}

    %____________________________________________________________________________________
    % Professional Membership
    \section{\mysidestyle Professional \\Activities}
    \begin{footnotesize} 
    {\bf Membership}\\
    British Society for Geomorphology; International Association of Sedimentologists (IAS); American Geophysical Union (AGU); Coastal Zone Network (COZONE); The Challenger Society for Marine Science.

    {\bf Journal Review} \\
 	Arctic; Continental Shelf Research; Earth Surface Processes and Landforms; Geo-Marine Letters; Geophysical Research Letters; Journal of Hydraulic Engineering; Journal of Mountain Science; Journal of Sedimentary Research; Marine Geology; Sedimentology; Sedimentary Geology; Water Resources Research.
     \end{footnotesize}

%    %____________________________________________________________________________________
%    % Conferences Organised
    \begin{footnotesize} 
    {\bf Conferences Organised}\\
	On the organising committee for:
	\begin{list1}
	\item[1] {\sl The Quaternary Research Association's 4th International Postgraduate Symposium}, hosted by the School of Geography at the University of Plymouth 31st August - 2nd September 2005.\\
	\item[2] {\sl Young Coastal Scientist and Engineers Conference, 2007} (YCSEC 2007)  hosted by the School of Geography at the University of Plymouth 19-21 April 2007. \\
	\item[3] {\sl American Geophysical Union Fall Meeting}, December 2007: H60. Linking sediment supply, bed-sediment particle size, sediment transport, and bed morphology in fluvial, marine, and aeolian settings. Co-convened with David Rubin (USGS), David Topping (USGS), and Scott Wright (USGS).\\
	\item[4] {\sl American Geophysical Union Fall Meeting}, December 2013: EP010. Fluvial sediment budgets: Can we do better? Co-convened with David Topping (USGS), Paul Grams (USGS), and Susannah Erwin (USGS).\\
	\item[5] {\sl 2nd Multibeam in Rivers Workshop}, March 2015. Co-convened with Paul Grams (USGS), Matt Kaplinski (NAU) and Joe Wheaton (USU).	
	\end{list1}
        \end{footnotesize}

    %____________________________________________________________________________________
%    % Computer Skills
    \section{\mysidestyle Skills} 
    \begin{footnotesize}
    \begin{list1}
        \item[1] Community models: General Ocean Turbulence Model (GOTM, \url{http://www.gotm.net/index.php}); Simulating Waves Nearshore (SWAN; \url{http://www.swan.tudelft.nl/}); Simulating Waves 'til Shore (SWASH; \url{http://swash.sourceforge.net/features/features.htm}).\\
        \item[2] Linux. High performance and distributed computing.\\
        \item[3] Programming/Scripting: Python, BASH, Matlab (proficient); Cython, Kivy, Fortran (experienced); C, R (beginner).\\
        \item[4] Full UK driving licence. Arizona State driving licence. LANTRA sit-astride ATV qualification.\\
        \item[5] Other interests: instrument control, machine vision, \LaTeX\, GUI development, machine learning, spectral analysis, image analysis
    \end{list1}
     \end{footnotesize}

    %____________________________________________________________________________________
    %  invited lectures
 	\section{\mysidestyle Invited \\Talks}
       \begin{footnotesize}
 	\begin{list1}
 	\item[1] {\sl Slapton Research Seminar, Field Studies Council, Slapton Ley}, 4th December 2004. Talk entitled `A tale of two storms'.\\
 	\item[2] {\sl Slapton Research Seminar, Field Studies Council, Slapton Ley}, 18th November 2006. Talk entitled `A view from the beach'\\
 	\item[3] {\sl Centre for Coastal Dynamics and Engineering (C-CoDE)}, University of Plymouth, 6th December 2006.  Talk entitled `Field observations of morphological change and sediment dynamics from the nearshore of a gravel beach'\\
 	\item[4] {\sl Slapton Research Seminar, Field Studies Council, Slapton Ley}, 3rd November 2007. Talk entitled `A year in the life of Slapton Sands - but was it a typical year?' with Tom Deacon (SLFC).\\
 	\item[5] {\sl Lancaster University Environmental Imaging Network}, 20th May 2008. Talk entitled `Optical sensing of gravel sediment transport and characteristics: recent advances and future challenges'.\\
         \item[6] {\sl Coastal and Marine Geology, United States Geological Survey, Santa Cruz}, 28th January 2009. Talk entitled `Morphodynamics and sediment dynamics of a macrotidal gravel beach'.\\
         \item[7] {\sl Centre for Coastal Science and Engineering, University of Plymouth}, 17th February 2010. Talk entitled `Turbulence, Sediment Stratification and Altered Resuspension under Waves'.\\
         \item[8] {\sl Grand Canyon Monitoring and Research Center, Flagstaff, Arizona}, 27th February 2012. Talk entitled `Nearshore Sediment Transport Through the Looking Glass'.\\
         \item[9] {\sl British Geological Survey, Marine Geosciences Division, Edinburgh}, 13th July 2012. Talk entitled `Digital Grain Size'. \\
         \item[10] {\sl Multibeam in Rivers Summit, Utah State University, Logan, Utah}, February 2014. Talk entitled `Bed Sediment Classification Using High-Frequency Acoustic Backscatter'. \\
         \item[11] {\sl Glen Canyon Dam Adaptive Management Program Adaptive Management Work Group Meeting, Flagstaff, Arizona}, August 2014. Talk entitled `Measuring bed sediments for improved sediment budgets and physical habitat assessment'. \\
         \item[12] {\sl USGS Coastal and Marine Geology, Woods Hole, MA}, February 2015. Talk entitled `The Digital Grain Size Project: Past, Present and Future'. 

 	\end{list1}
        \end{footnotesize}

%\newpage
%    \section{\mysidestyle Cruises, Field \& \\Laboratory Experiments} 
%        \begin{footnotesize}

%    {\sl Groundwater and swash sediment transport experiments at Slapton Sands, UK, October 2004.} \\
%    Member of the science team helping Martin Austin, Loughborough University. Topographic surveying, sediment sampling, ADV and EMCM current meters.

%    {\sl Surveys of sediment transport, morphological change and clast burial at Cayeux Spit, France, February 2005.} \\
%    Member of the science team helping Jerome Curoy, Sussex University. Topographic surveying, sediment sampling, sediment tracing.

%    {\sl XSHORE experiments, conducted at Sennen, UK, summer 2005.} \\
%    Member of the science team from University of Plymouth led by Paul Russell. Topographic surveying, optical backscatter sensors, various current meters.

%    {\sl Measurements of beach hydrodynamics, video observations of sediment transport under shoaling waves, morphological change and sediment dynamics. Surveys at Slapton Sands, UK, October 2005, May 2006, throughout 2007.} \\
%    Conducting research for my PhD, setting up an ARGUS coastal video monitoring system, and implementing a long-term survey protocol for this beach which is still in operation. Topographic surveying, sediment sampling, bed sediment cameras, video measurements, wave resistance wires, ADV and EMCM current meters.

%    {\sl Measurements of beach hydrodynamics, sediment transport, and morphological change. ECORS-2008 at Truc Vert, France, March 2008.} \\
%    Member of an international science team led by University of Bordeaux consisting of individuals from the universities and research institutes in the UK, USA, France, Australia, and New Zealand. Topographic surveying, current meters, optical backscatter sensors. Webpage: \url{http://ecors.epoc.u-bordeaux1.fr/index.php?page=2}

%    {\sl WHISSP, Spring 2008.} \\
%    Member of the science team from University of Plymouth led by Jon Williams, carrying out topographic surveys using an ATV equipped with RTK-GPS, setting up ARGUS coastal video monitoring systems, and implementing a long-term survey protocol for the beaches of north Cornwall which is still in operation. Webpage: \url{http://www.perc.plymouth.ac.uk/whissp/}

%    {\sl BARDEX, The Netherlands, summer 2008. EU-funded 6-week laboratory experiment at the Deltaflume large wave test facility.} \\
%    Member of an international science team from the universities and research institutes in the UK, The Netherlands, Portugal, and Australia. Topographic surveying, bed sediment cameras, video measurements, optical backscatter sensors, various sonars and current meters. Webpage: \url{http://www.connectedwaters.unsw.edu.au/news/bringingthebeach.html}

%    {\sl Video observations of sediment transport under shoaling waves at Slapton Sands, UK, summer 2008.} \\
%    Conducting research with Dr Alex Nimmo-Smith using a rig with current meters and two stereo pairs of high resolution underwater cameras.

%    {\sl Measurements of shelf hydrodynamics and sediment dynamics. Santa Cruz Seafloor Observatory, various times 2008 - 2009.} \\
%    Member of the science team led by Dave Rubin consisting of individuals from from USGS Santa Cruz deploying and maintaining two benthic tripods just offshore from Santa Cruz, on the R/V Snavely. Various sonars and point current meters, bed sediment cameras, optical backscatter sensors, LISST, CTD, ADCP, PCADP.

%    {\sl Colorado River cruise, summer 2009.} \\
%    Member of the science team led by Ted Melis consisting of individuals from USGS Santa Cruz, USGS Flagstaff, USGS Sacramento, Utah State University, and University of California Santa Barbara. Sidescan sonar, bed sediment cameras, fathometer.

%    {\sl Beaches of Elwha, Washington State, August 2009.} \\
%    Member of the science team led by Jon Warrick, USGS Santa Cruz. Topographic surveying and bed sediment cameras. Webpage: \url{http://walrus.wr.usgs.gov/elwha/}. 

%    {\sl Installation of HF-Radar systems at Perranporth and Pendeen, North Cornwall, various times in 2010. These systems are designed to provide large-scale measurements of waves and surface currents over the Celtic Sea.} \\
%    Member of the science team led by Daniel Conley, Plymouth University.

%    {\sl Cruises to L4, English Channel, various times in 2010. Measurements of column turbulence and the onset of stratification.} \\
%    Member of the science team led by Jaimie Cross, Plymouth University, on the R/V Quest. Wire-walker, microstructure profiler, holographic particle imaging system.

%    {\sl TSSAR Waves (Turbulence, Sediment Stratification and Altered Resuspension under Waves) experiment, conducted at Praa Sands, UK, May 2011.} \\
%    Lead member of an international science team consisting of individuals from the Universities of Plymouth (UK) and USGS (USA). Topographic surveying, bed sediment cameras, video measurements, optical backscatter sensors, various sonars and current meters, optical and fibre-optical backscatter sensors, holographic particle imaging system, acoustic backscatter system. Webpage: \url{http://www.research.plymouth.ac.uk/tssar_waves/}

%    {\sl Oscillating grid turbulence tank laboratory experiments, University of Plymouth, 2011 and ongoing.} \\
%    Shaking grid experiments started. Measuring suspended sediment and turbulence using fibre-optical backscatter sensors and Vectrino high-frequency ADVs.

%    {\sl BEST (Beach Sediment Transport) experiment, conducted at Perranporth, UK, October 2011.} \\
%    Member of an international science team led by Gerd Masselink consisting of individuals from the universities of Plymouth (UK), Delaware (USA), and New South Wales (Australia). Topographic surveying, optical backscatter sensors, ultrasonic bed-level sensors, arious current meters, fibre-optical backscatter sensors, conductivity concentration profiler. Webpage: \url{http://www.bbc.co.uk/news/uk-15270548}

%{\sl DRIBS (Dynamic Rips and Implications for Beach Safety) experiment, conducted at Boscombe, Bournemouth, UK, October 2012.} \\
%    Member of a science team led by Tim Scott. Measurements of rip currrnts using drifters (drogues), videoed dye releases, and acoustic measurements of flow fields.


%\end{footnotesize}

    %____________________________________________________________________________________
    % Referees
     %\section{\mysidestyle Referees} 
 
     %{\sl Available on request.}


%________________________________________________________________________________________
\end{resume}
\end{document}

%________________________________________________________________________________________
% EOF
